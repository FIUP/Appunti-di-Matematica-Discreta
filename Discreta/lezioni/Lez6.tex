\section{Sesta lezione: Grafi planari}
Un grafo non orientato si dice \emph{planare} se lo si può disegnare sul piano senza intersezioni tra gli archi. \\
Tale disegno viene detto \emph{rappresentazione piana} del grafo.\\

\noindent
Come stabilire se un grafo è planare? \\
Problema polinomiale, anche se l'algoritmo è complicato. Ne daremo una versione semplificata, chiamata il \emph{metodo del cerchio e delle corde}.

\subsection{Minori}
Contrazione di un arco u,v. Identifico u, v in un unico vertice. Arco u,v è diventato
un cappio ( e viene usualmente rimosso.) \\
G'(V',E') è minore di G(V,E) se G' è ottenuto da G tramite:
\begin{itemize}
\item contrazione di archi
\item rimozione di archi
\item rimozione di vertici isolati
\end{itemize}
\textbf{Teorema:} G planare, G' minore di G, allora G' planare. \\
Ognuna delle 3 operazioni applicata ad una rappresentazione piana di G, mantiene piana la rappresentazione.

\subsection{Teorema di Kuratowski}
Un grafo è planare se e solo se non contiene $K\ped{3,3}$ o $K_5$ come minore.\\

\noindent
\subsection{Disegni equivalenti per grafi planari:} Dato un grafo planare, ci sono molte sue rappresentazioni piane, tutte equivalenti. \\
Ogni rappresentazione piana divide il piano in \emph{regioni} o \textbf{facce}. Indichiamo con \emph{r} il numero di queste
regioni. Il numero r dipende solo dal grafo, non dalla particolare rappresentazione piana disegnata. 

\subsubsection{Metodo del cerchio e delle corde}
Dato un grafo G(V,E) non orientato:
\begin{enumerate}
\item trovare, se esiste, un circuito di lunghezza massima ( che
contenga tutti i vertici del grafo, lo chiameremo circuito
Hamiltoniano)
\item disegnare questo circuito come un grande cerchio
\item scrivere l'elenco degli archi del grafo che non sono
contenuti nel circuito; li chiamiamo corde e li inseriamo
internamente o esternamente al cerchio, cercando di
evitare gli incroci, seguendo i passi 4 e 5
\item scegliamo una corda e inseriamola, ad esempio,
internamente al cerchio, togliendola dall'elenco;
\item Tutte le corde che la incrociano debbono essere inserite
esternamente. Se la rappresentazione e' piana,
proseguiamo. Altrimenti il grafo non e' planare.
\end{enumerate}
\emph{Note:}
\begin{itemize}
\item Attenzione all'ordine con il quale si inseriscono le corde:
non possiamo fare scelte, solo la prima corda può essere
posta internamente o esternamente a nostra scelta.
\item Questa procedura non è sempre applicabile; in
particolare, non è applicabile se il grafo non contiene un
circuito hamiltoniano.
\end{itemize}

\subsubsection{Esempi}

\subsection{Formula di Eulero}
Se G è un grafo non orientato, connesso e planare, allora ogni sua rappresentazione piana ha r facce con:
\[r = e -v + 2\]
Con \(v = |V| = \) numero di vertici,
\(e = |E| =\) numero di archi,
\(r =\) numero di facce (regioni). \\

\noindent
\textbf{Dimostrazione:}

\subsection{Politopi convessi}
\paragraph{Formula di Eulero per i politopi}
Sia P un politopo convesso nello spazio tridimensionale. \\
Sia \(v =\) numero dei vertici,
\(e =\) numero degli archi,
\(r =\) numero delle facce,
allora vale la relazione \(r = e - v + 2\) . \\
\textit{Esempio:} un cubo ha \(r=6, e=12 v=8 \Rightarrow e- v+2=12-8+2=6=r\)
Ogni politopo si può trasformare in un grafo connesso e planare.

\subsubsection{Corollari}
\paragraph{Corollario 1:} Se G(V,E) è un grafo semplice e planare, allora \(e \leq 3v - 6\). \\
\textbf{Dimostrazione:}

\paragraph{Corollario 2:} Se G(V,E) è un grafo planare semplice, connesso e bipartito, allora \(e \leq 2v - 4\).\\
\textbf{Dimostrazione:}

\paragraph{Corollario 3:} Se G(V,E) planare, semplice e G contiene un vertice di grado \(<= 5\) non posso applicare la formula \(e<=3v-6\) perchè se ogni nodo ha grado $>5$ (cioè $\geq 6$) la formula \(\sum \ped{v \in V} d(v) = 2 |E| \) crea contraddizione, infatti: \(6v = 2|E|\) ossia \(3v = e\). 

\subsection{Problema dei 4 colori}
\textbf{Problema:} quanti colori sono necessari per colorare i paesi di una
mappa in modo tale che paesi adiacenti abbiano colori diversi?
\begin{center}
Mappa \\
\(\Downarrow \) \\
Grafo planare, ad ogni paese corrisponde un vertice.
\end{center}
\textbf{Problema:} quanti colori sono necessari per colorare i vertici di un grafo
planare, in modo che vertici adiacenti abbiano colori diversi? \\

\noindent
\textbf{Congettura dei 4 colori, dimostrata da Appel e Haken (1976):} tutti i grafi planari sono 4-colorabili.


\newpage
