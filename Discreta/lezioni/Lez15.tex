\section{Quindicesima lezione: Esercizi}

\paragraph{Es.1} %esercizio 2
Trovare una relazione di ricorrenza per valutare il numero di coppie di conigli dopo n mesi
se:
\begin{itemize}
    \item inizialmente vi è solo una coppia di conigli appena nati
    \item ogni mese, ogni coppia di conigli che ha più di un mese genera una nuova coppia di
conigli
    \item nessun coniglio muore
\end{itemize}

\paragraph{Es.2} %esercizio 4
\begin{itemize}
    \item Trovare una relazione di ricorrenza per il numero di sequenze quaternarie (cioè ad elementi in {0, 1, 2, 3}) di lunghezza n
     \item con almeno un 1, con il primo 1 che precede il primo 0 (se la sequenza contiene almeno uno 0)
\end{itemize}    

\paragraph{Es.3} Trovare la relazione di ricorrenza per il numero delle sequenze ternarie
(0 1 2) che non contengono la sequenza "012"

\paragraph{Es.4} Quante sono le scacchiere contenute in una scacchiera n x n?


\newpage