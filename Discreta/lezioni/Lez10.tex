\section{Decima lezione:}
Coefficenti binomiali: \[C(n,k) =  \binom{n}{k} \]
\\
\[ \binom{n}{k} = \frac{n!}{K!(n-k)!}\] se \( 0 \leq k \leq n\) interi.
\subsection{Triangolo di Pascal/Tartaglia}

\subsection{Identità binomiali} 
Verieficati in due modi diversi: 
\begin{itemize}
    \item espansione binomiale 
    \item significato combinatorio
\end{itemize}

\begin{enumerate}
    \item 
    \item 
    \item 
\end{enumerate}

\subsection{Esercizi}

\paragraph{Es. 1} Dimostrarze che la seguente identità binomiale vale per ogni n naturale,
usando l'interpretazione combinatoria dei coefficienti binomiali.
Verificarla anche tramite l'espansione algebrica dei coefficienti binomiali.

\paragraph{Es. 2} Dimostrare la seguente identità binomiale usando l'interpretazione combinatoria dei
coefficienti binomiali, dove m, n ed r sono numeri naturali, con \(0 \leq r \leq \min (m,n)\)

\paragraph{Es. 3} %esercizio 6
Risolvere la seguente equazione binomiale per n naturale, trovando tutte le sue soluzioni.

\paragraph{Es. 4} Dimostrate con metodi di conteggio oppure identità già viste le seguenti identità binomiali:

\paragraph{Es. 5} %eserc 3
Dimostrare la seguente identità binomiale, per n e k naturali con $0 \leq k \leq n$.

\paragraph{Es. 6}%eserc 5
Valutare le seguenti espressioni per n naturale (n diverso da zero per la prima):


\newpage
