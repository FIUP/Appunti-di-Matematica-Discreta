
\section{Seconda lezione: Grafi orientati e non orientati}
\subsection{Grafi non orientati}
\textsc{definizione:} un grafo (non orientato) è una coppia G(V,E) dove:
\begin{itemize}
    \item \( V={v_1, ...,v_n} \) è un insieme finito di vertici (nodi) rappresentati da punti sul piano.
    \item E è un insieme di archi che sono coppie \emph{non} ordinate di vertici, detti spigoli.
\end{itemize}
Un grafo non orientato è anche detto \emph{grafo semplice} ossia nell'insieme delle coppie non ci sono né cappi né archi(spigoli) paralleli.\\

\textit{Ex:} \par
\( V = \{ v\ped{1}, .. , v\ped{8} \} \) \par
\( E = \{ (v\ped{1}, v\ped{3}), (v\ped{1}, v\ped{7}), (v\ped{2}, v\ped{3}), .. \} \) \\

Definizioni:
\begin{itemize}
    \item Estremi di un arco
    \item \emph{Arco incidente} nei suoi estremi, cioè collega due vertici
    \item \emph{Vertici adiacenti} (nonadiacenti) : sono gli estremi di un arco
    \item \emph{Grado del vertice v} (si indica con $d(v)$): numero di volte in cui v è estremo di un arco. \(d(v3)=3, d(v8)=0 \)
\end{itemize}

\paragraph{Due proprietà fondamentali} Per ogni grafo G(V, E) vale la seguente relazione:

    \begin{equation}
        \sum \ped{v \in V} d(v) = 2 \mid E \mid 
    \end{equation} \\

    Poichè ogni arco ha 2 estremi, contribuisce due volte alla somma dei gradi. \\
    \paragraph{Corollario:} Dato G(V,E) non orientato, alcuni gradi sono dispari altri pari. Il numero di vertici di grado dispari è sempre pari. Ossia:

    \begin{equation}
    \sum{v \in V} d(v) = \sum{v \in V_d} d(v) + \sum{v \in V_p} d(v)
    \end{equation} \\

    \noindent
    \textit{NdA:} la sommatoria dei gradi di tutti i vertici è un multiplo di 2, poichè essa è equivalente a \( 2 \mid E \mid\). Ne consegue che se dividiamo i vertici per grado dispari e pari entrambi i risultati delle sommatorie dovranno essere pari (Ex: $2 +4$ è pari , $5+2$ è dispari). \\
    La prima sommatoria è fatta da vertici con grado dispari che possono essere per esempio, $3,5, \dots$ \\ 
    Quando vengono sommati due numeri dispari se ne ottiene uno pari, ma se invece sommassimo 3 numeri dispari avremmo un risultato dispari. (Ex: $3+3$ è pari, $3+3+3$ è dispari). Nella seconda sommatoria non ci interessa sapere quant'è la quantità di gradi che si hanno effettivamente perchè ne risulterà sempre una somma pari.\\

    Ogni grafo semplice ha al \emph{massimo} \( \left( \begin{array}{c} n \\ 2 \end{array} \right) = \frac{n(n-1)}{2} \) archi dove \( n = \mid V \mid \). \par
    \( \left( \begin{array}{c} n \\ 2 \end{array} \right) \) è il numero di coppie non ordinate di n elementi. \\
    Suppondendo di avere un grafo composto da \(V = n\), il numero minimo di archi è 0, mentre quello massimo è $\frac{n(n-1)}{2}$.\\
    Ne deriva che:
    \begin{equation}
    0 \leq \mid E \mid \leq \frac{n(n-1)}{2}
    \end{equation}
    Nei grafi bipartiti il numero \emph{minimo} è 0 mentre quello massimo è dato dal prodotto fra la quantità dei vertici dei due sottoinsiemi:
    \begin{equation}
    \mid V1 \mid * \mid V2 \mid
    \end{equation}
    Se vale \( \mid V1 \mid * \mid V2 \mid =E \) allora ho un grafo bipartito completo.

\subsubsection{Terminologia}

\textbf{Cammino o path:} corrisponde a una sequenza di vertici distinti, dove ogni coppia di vertici consecutivi nel cammino è collegata da un arco. \\
\textbf{Circuito o ciclo:} è un particolare cammino nel quale il primo e l'ultimo vertice sono coincidenti. \\
\textbf{Lunghezza:} numero di archi.\\
\textbf{Pari, dispari:} parità della lunghezza, cioè se la quantità del numero di archi è pari o dispari. \\
\textbf{Connesso:} per ogni coppia di vertici esiste un
cammino che li collega \\
\textbf{Percorso:} sequenza di vertici \textsc{non necessariamente} distinti, dove ogni coppia di vertici
consecutivi nel percorso è collegata da un arco. (Cioè è una maniera di attraversare parte del
grafo, possibilmente passando più di una volta per lo stesso vertice o arco). Un percorso è \emph{chiuso}
se le estremità coincidono.

\paragraph{Teorema:} Dato un percorso con estremità \( v\ped{1}, v\ped{n} \) esiste un cammino con estremità \( v\ped{1}, v\ped{n} \).
\paragraph{Teorema:} Un percorso chiuso di lunghezza dispari contiene un ciclo di lunghezza dispari. \\
Se gli unici vertici coincidenti sono le estremità allora il percorso è un ciclo dispari. Altrimenti il percorso si partiziona in due percorsi di lunghezza minore, uno pari e l'altro dispari.

\subsubsection{Alcuni grafi} 
\textbf{Grafo completo:} ogni coppia di vertici è adiacente(collegati da un arco). E' denominato $K_n$ (n è il numero di vertici) \\
\textbf{Bipartito:} i vertici sono partizionabili in due sottoinsiemi, U e W, e ogni arco è incidente in un vertice di U e uno di W. \\
\textbf{Grafo bipartito completo} K\ped{n1},\ped{n2} Grafo bipartito, le due parti della bipartizione hanno \( \ped{n1}, \ped{n2} \) vertici rispettivamente ed ogni vertice di una parte è adiacente a tutti i vertici della seconda parte. \\
\textbf{Foresta:} Grafo senza cicli ($=$aciclico) \\
\textbf{Albero:} Foresta connessa. \\
Un grafo è K-regolare se ogni nodo ha grado k. I grafi 3-regolari si chiamano \textsc{grafi cubici}. \\

\noindent
\textbf{Disegnare un grafo:} un "disegno" di un grafo ne illustra le proprietà. Ma i "disegni" dello stesso grafo sono molteplici. Noi
diremo che due grafi sono \emph{isomorfi} se uno è il disegno dell'altro. \\
Formalmente, G(V,E) e G’(V’, E’) sono isomorfi se esiste una biezione F: che porta V in V’ e che preserva le
adiacenze. Cioè uv è un arco di G se e solo se f(u)f(v) è un arco di G’. \\
\textbf{Multigrafi:} finora abbiamo visto grafi semplici, perchè fra ogni coppia di vertici c'è al massimo un arco. In generale possono esserci archi paralleli, cioè archi che hanno le stesse estremità e cappi: Archi
con le estremità coincidenti. Tali grafi si dicono multigrafi. \\
\textit{Grado di un nodo:} I cappi contribuiscono 2 al grado dell'unica estremità.\\
\textbf{Sottografi:} un sottografo di un grafo G(V,E) è un grafo G'(V', E') con V'$\subseteq$ V e E'$\subseteq$ E. \\
Dato G(V,E) un suo sottografo G'(V',E') si dice \textsc{indotto} da V' se E' è il sottoinsieme di archi in E con entrambe le estremità in V'. Cioè per ogni arco (u,v) $\in$ E, se u $\in$ V' e v $\in$ V' allora (u,v) $\in$ E'. 

\subsection{Grafi orientati}
\textsc{Definizione:} un grafo orientato è una coppia D(V,A), dove: \\
\( V={ v\ped{1}, ...,v\ped{n}} \) è un insieme finito di nodi o vertici. \\
L'insieme A di archi è un insieme di coppie \emph{ordinate} di vertici. \\
Un arco ha un senso di percorrenza(che si indica con una freccia), dalla testa alla coda. \\

Nodo iniziale e finale di uno arco $=$ coda e testa dell'arco
\begin{itemize}
\item \textbf{Archi paralleli:} hanno la stessa testa e la stessa coda.
\item \textbf{Cappio:} la testa e la coda coincidono.
\end{itemize}
\textbf{Grafo orientato Semplice:} non ci sono cappi e nemmeno archi paralleli.
\textbf{Multigrafo orientato:} ci sono cappi e archi paralleli. 

\subsubsection{Terminologia}
\textbf{Cammino orientato:} sequenza di nodi distinti, dove ogni coppia di nodi consecutivi nel cammino è
collegata da un arco (nodo iniziale, nodo finale) \\
\textbf{Lunghezza del cammino:} numero di archi nel cammino. \\
Un grafo orientato è \textit{fortemente connesso} se per ogni coppia di nodi $u$, $v$ esiste un cammino orientato da $u$ a $v$ ed un cammino orientato da $v$ a $u$. \\
\textbf{Circuito:} cammino nel quale il primo nodo coincide con l'ultimo. \\
\textbf{Lunghezza del circuito:} numero di archi. \\
\textbf{Parità del circuito:} parità della lunghezza.

\subsubsection{Proprietà dei grafi orientati}
\textbf{\textit{Semplice} massimo numero di archi}: numero di coppie ORDINATE di \(n=\mid V \mid \) elementi. \\
\textbf{In-degree(v)} $=$ Grado entrante di un nodo v: numero di archi entranti in v \(\Rightarrow d\ap{in}(v)\) \\
\textbf{Out-degree(v)} $=$ Grado uscente di un nodo v: numero di archi uscenti da v \( \Rightarrow d\ap{out}(v) \)\\

Per ogni grafo orientato D(V,A) vale:
 \[ \mid A \mid = \sum{v \in V} d\ap{in}(v) = \sum{_v \in V} d\ap{out}(v) \] 

\paragraph{Teorema} In ogni grafo orientato D(V,A), sono uguali tra loro:
\begin{itemize}
\item la somma dei gradi uscenti dei nodi
\item la somma dei gradi entranti dei nodi
\item il numero di archi del grafo
\end{itemize}

\begin{equation}
    \sum{v \in V} d\ap{out}(v) = \sum{v \in V} d\ap{in}(v) = \mid A \mid 
\end{equation}

Un arco uv un contribuisce 1 a d\ap{out}(v) e 1 a d\ap{in}(v)

\paragraph{Tornei:} Un \textit{torneo} è un grafo orientato in cui per ogni coppia di vertici esiste \textsc{esattamente} uno degli archi uv , uv. \\
(Si chiamano tornei (tournaments) perche indicano gli esiti di un torneo e il senso di ogni arco indica l'esito dell'incontro fra i giocatori rappresentati dai suoi estremi).

\subsubsection{Rappresentazione di grafi mediante matrici}
\begin{enumerate}
\item \textsc{Matrice di incidenza vertici/archi}
\item \textsc{Matrice di incidenza nodi/archi}
\item \textsc{Matrice di adiacenza vertici/vertici o nodi/nodi}
\item List of edges, linked adjacency list, forward star, backward star ect..
\end{enumerate}

\newpage