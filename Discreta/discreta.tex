\documentclass[12pt,a4paper]{article}

\usepackage[english,italian]{babel}
\usepackage[utf8]{inputenc}
\usepackage[T1]{fontenc}

\usepackage{hyperref}
\hypersetup{hidelinks}

\usepackage{graphicx} %img/media
\usepackage{enumitem} %lists
\usepackage{listings} %coding environment 
\usepackage{mathtools} %math package
\usepackage{amssymb}
\usepackage{amsmath}
\usepackage{amsthm}


\begin{document}
    \author{Sara Righetto}
    \title{Appunti di Matematica Discreta}
   % \includegraphics[width=0.50\textwidth]{img/logo.png}

    \maketitle

    \newpage
    \tableofcontents %indice

    \newpage

    \begin{abstract} 
        La matematica discreta, è lo studio di strutture matematiche che sono fondamentalmente discrete, nel senso che non
    richiedono il concetto di continuità che voi vedete nei corsi di analisi. \par
    La maggior parte degli oggetti studiati nella matematica discreta sono insiemi numerabili come i numeri interi o
    insiemi finiti, come gli interi da 1 a k. \par
    Altri oggetti importanti sono i grafi, che modellano reti di connessione. \\

        La matematica discreta è diventata famosa per le sue applicazioni in informatica. I concetti e le notazioni della matematica discreta sono utili per lo studio o la modellazione di oggetti o problemi negli algoritmi informatici e nei linguaggi di programmazione.
    \end{abstract}

\begin{center}
    \textit{Argomenti:}
    \begin{enumerate}
        \item Teoria dei grafi 
        \item Metodi di conteggio
        \item Relazioni di ricorrenza
    \end{enumerate}
\end{center}

    \newpage


%LEZIONI

    \section{Prima lezione: Introduzione}

    \subsection{Teoria dei grafi}
        Grafo = G(V,E) \\
        \textbf{Planarità di un circuito stampato:} 
Nei circuiti stampati, le componenti elettroniche sono collegate da piste conduttrici stampate su una tavoletta
isolante; queste piste non si possono incrociare.\\

\textsc{Problema del postino cinese} \\
Un postino deve consegnare la posta nelle seguenti strade. Può partire da A e tornare in A senza percorrere due volte
la medesima strada ? \\
Equivalente: Disegnare la seguente figura senza sollevare la penna dal foglio (partendo e tornando in un punto
arbitrario) \\

\textsc{Spelling checker} \\
Cercare una parola nel dizionario. Zucca: si ricorre all'uso degli alberi binari \\

\textsc{Torneo ad eliminazione} \\
16 Giocatori: Torneo ad eliminazione diretta. Anche in questo caso si usano gli alberi binari.

\subsection{Metodi di conteggio} 
Contare un numero (finito) di oggetti non è sempre facile ma è un problema che si pone spesso in
informatica: per esempio contare le operazioni fatte da un algoritmo per misurarne l'efficienza. \\
\begin{center}
\textsc{Metodi di conteggio $=$ Enumerazione $=$ Calcolo combinatorio}
\end{center} 

\textit{Esempi:} 
\begin{itemize}
        \item Numero di parole con un fissato alfabeto
\item Numero di sequenze binarie, numero di sottoinsiemi di n elementi
\item Numero di soluzioni intere di una data equazione lineare
\item Numero di targhe automobilistiche
\item Probabilità discreta: Quale è la probabilità che 4 carte diano un poker? (d'assi?)
\end{itemize}

\subsection{Relazioni di ricorrenza}
Spesso è difficile calcolare direttamente una quantità (come il numero \( K\ped{n}\)
di sottoinsiemi di n
elementi con una data proprietà). E' più semplice trovare una relazione di ricorrenza che
determina \( K\ped{n} \) in funzione di \( K\ped{n-1} \). \\
\begin{center}
\textsc{Relazioni di ricorrenza $=$ Equazioni alle differenze finite}
\end{center}

Problemi ricorsivi:
\begin{enumerate}
\item Calcolo del determinante di una matrice quadrata
\item Numeri di Fibonacci
\item Calcolo di n!
\item Torre di Hanoi
\end{enumerate} 

\texttt{Esempio 1.1:} \par
    \(a\ped{n} = a\ped{n-1} * n \rightarrow \) \textsc{relazione di ricorrenza} \par
    \(a\ped{0} = 1 \rightarrow \) \textsc{condizione iniziale} \\

    \( a\ped{n} = n! \rightarrow \) \textsc{soluzione} \\

\texttt{Esempio 1.2:} \\
\textbf{Evoluzione di un capitale investito in banca:}  All'anno 0 investo 1000EUR. Il mio investimento dà un interesse del 4 \% annuo. Dopo n anni quanto e' il mio capitale? \\

\(a\ped{0} = 1000 \) \par
\(a\ped{1} = 1000 + \frac{4}{100} * 1000 = 1.04 * a\ped{0} \) \par
\(a\ped{n} = (1.04) * a\ped{n-1} \rightarrow \) \textsc{relazione di ricorrenza} \par

\(a\ped{2} = 1.04 * a\ped{1} = (1.04)^2 * a\ped{0} \) \par

\(a\ped{n} = (1.04)^n * a\ped{0} \rightarrow \) \textsc{soluzione} 
\\

\texttt{Esempio 2:} \\
\textbf{Calcolare il massimo di una stringa ($x_1$, .. , $x_n$ ) di n numeri} \\
$a_n$ $=$ numero di confronti necessari per calcolare il massimo di una stringa di lunghezza n \\

\( a\ped{n} = a\ped{n/2} + a\ped{n/2} +1 \rightarrow \) \textsc{relazione di ricorrenza} \par

\( a\ped{2} = 1 \rightarrow \) \textsc{soluzione} 

\newpage



\section{Seconda lezione: Grafi orientati e non orientati}
\subsection{Grafi non orientati}
\textsc{definizione:} un grafo (non orientato) è una coppia G(V,E) dove:
\begin{itemize}
    \item \( V={v_1, ...,v_n} \) è un insieme finito di vertici (nodi) rappresentati da punti sul piano.
    \item E è un insieme di archi che sono coppie \emph{non} ordinate di vertici, detti spigoli.
\end{itemize}
Un grafo non orientato è anche detto \emph{grafo semplice} ossia nell'insieme delle coppie non ci sono né cappi né archi(spigoli) paralleli.\\

\textit{Ex:} \par
\( V = \{ v\ped{1}, .. , v\ped{8} \} \) \par
\( E = \{ (v\ped{1}, v\ped{3}), (v\ped{1}, v\ped{7}), (v\ped{2}, v\ped{3}), .. \} \) \\

Definizioni:
\begin{itemize}
    \item Estremi di un arco
    \item \emph{Arco incidente} nei suoi estremi, cioè collega due vertici
    \item \emph{Vertici adiacenti} (nonadiacenti) : sono gli estremi di un arco
    \item \emph{Grado del vertice v} (si indica con $d(v)$): numero di volte in cui v è estremo di un arco. \(d(v3)=3, d(v8)=0 \)
\end{itemize}

\paragraph{Due proprietà fondamentali} Per ogni grafo G(V, E) vale la seguente relazione:

    \begin{equation}
        \sum \ped{v \in V} d(v) = 2 \mid E \mid 
    \end{equation} \\

    Poichè ogni arco ha 2 estremi, contribuisce due volte alla somma dei gradi. \\
    \paragraph{Corollario:} Dato G(V,E) non orientato, alcuni gradi sono dispari altri pari. Il numero di vertici di grado dispari è sempre pari. Ossia:

    \begin{equation}
    \sum{v \in V} d(v) = \sum{v \in V_d} d(v) + \sum{v \in V_p} d(v)
    \end{equation} \\

    \noindent
    \textit{NdA:} la sommatoria dei gradi di tutti i vertici è un multiplo di 2, poichè essa è equivalente a \( 2 \mid E \mid\). Ne consegue che se dividiamo i vertici per grado dispari e pari entrambi i risultati delle sommatorie dovranno essere pari (Ex: $2 +4$ è pari , $5+2$ è dispari). \\
    La prima sommatoria è fatta da vertici con grado dispari che possono essere per esempio, $3,5, \dots$ \\ 
    Quando vengono sommati due numeri dispari se ne ottiene uno pari, ma se invece sommassimo 3 numeri dispari avremmo un risultato dispari. (Ex: $3+3$ è pari, $3+3+3$ è dispari). Nella seconda sommatoria non ci interessa sapere quant'è la quantità di gradi che si hanno effettivamente perchè ne risulterà sempre una somma pari.\\

    Ogni grafo semplice ha al \emph{massimo} \( \left( \begin{array}{c} n \\ 2 \end{array} \right) = \frac{n(n-1)}{2} \) archi dove \( n = \mid V \mid \). \par
    \( \left( \begin{array}{c} n \\ 2 \end{array} \right) \) è il numero di coppie non ordinate di n elementi. \\
    Suppondendo di avere un grafo composto da \(V = n\), il numero minimo di archi è 0, mentre quello massimo è $\frac{n(n-1)}{2}$.\\
    Ne deriva che:
    \begin{equation}
    0 \leq \mid E \mid \leq \frac{n(n-1)}{2}
    \end{equation}
    Nei grafi bipartiti il numero \emph{minimo} è 0 mentre quello massimo è dato dal prodotto fra la quantità dei vertici dei due sottoinsiemi:
    \begin{equation}
    \mid V1 \mid * \mid V2 \mid
    \end{equation}
    Se vale \( \mid V1 \mid * \mid V2 \mid =E \) allora ho un grafo bipartito completo.

\subsubsection{Terminologia}

\textbf{Cammino o path:} corrisponde a una sequenza di vertici distinti, dove ogni coppia di vertici consecutivi nel cammino è collegata da un arco. \\
\textbf{Circuito o ciclo:} è un particolare cammino nel quale il primo e l'ultimo vertice sono coincidenti. \\
\textbf{Lunghezza:} numero di archi.\\
\textbf{Pari, dispari:} parità della lunghezza, cioè se la quantità del numero di archi è pari o dispari. \\
\textbf{Connesso:} per ogni coppia di vertici esiste un
cammino che li collega \\
\textbf{Percorso:} sequenza di vertici \textsc{non necessariamente} distinti, dove ogni coppia di vertici
consecutivi nel percorso è collegata da un arco. (Cioè è una maniera di attraversare parte del
grafo, possibilmente passando più di una volta per lo stesso vertice o arco). Un percorso è \emph{chiuso}
se le estremità coincidono.

\paragraph{Teorema:} Dato un percorso con estremità \( v\ped{1}, v\ped{n} \) esiste un cammino con estremità \( v\ped{1}, v\ped{n} \).
\paragraph{Teorema:} Un percorso chiuso di lunghezza dispari contiene un ciclo di lunghezza dispari. \\
Se gli unici vertici coincidenti sono le estremità allora il percorso è un ciclo dispari. Altrimenti il percorso si partiziona in due percorsi di lunghezza minore, uno pari e l'altro dispari.

\subsubsection{Alcuni grafi} 
\textbf{Grafo completo:} ogni coppia di vertici è adiacente(collegati da un arco). E' denominato $K_n$ (n è il numero di vertici) \\
\textbf{Bipartito:} i vertici sono partizionabili in due sottoinsiemi, U e W, e ogni arco è incidente in un vertice di U e uno di W. \\
\textbf{Grafo bipartito completo} K\ped{n1},\ped{n2} Grafo bipartito, le due parti della bipartizione hanno \( \ped{n1}, \ped{n2} \) vertici rispettivamente ed ogni vertice di una parte è adiacente a tutti i vertici della seconda parte. \\
\textbf{Foresta:} Grafo senza cicli ($=$aciclico) \\
\textbf{Albero:} Foresta connessa. \\
Un grafo è K-regolare se ogni nodo ha grado k. I grafi 3-regolari si chiamano \textsc{grafi cubici}. \\

\noindent
\textbf{Disegnare un grafo:} un "disegno" di un grafo ne illustra le proprietà. Ma i "disegni" dello stesso grafo sono molteplici. Noi
diremo che due grafi sono \emph{isomorfi} se uno è il disegno dell'altro. \\
Formalmente, G(V,E) e G’(V’, E’) sono isomorfi se esiste una biezione F: che porta V in V’ e che preserva le
adiacenze. Cioè uv è un arco di G se e solo se f(u)f(v) è un arco di G’. \\
\textbf{Multigrafi:} finora abbiamo visto grafi semplici, perchè fra ogni coppia di vertici c'è al massimo un arco. In generale possono esserci archi paralleli, cioè archi che hanno le stesse estremità e cappi: Archi
con le estremità coincidenti. Tali grafi si dicono multigrafi. \\
\textit{Grado di un nodo:} I cappi contribuiscono 2 al grado dell'unica estremità.\\
\textbf{Sottografi:} un sottografo di un grafo G(V,E) è un grafo G'(V', E') con V'$\subseteq$ V e E'$\subseteq$ E. \\
Dato G(V,E) un suo sottografo G'(V',E') si dice \textsc{indotto} da V' se E' è il sottoinsieme di archi in E con entrambe le estremità in V'. Cioè per ogni arco (u,v) $\in$ E, se u $\in$ V' e v $\in$ V' allora (u,v) $\in$ E'. 

\subsection{Grafi orientati}
\textsc{Definizione:} un grafo orientato è una coppia D(V,A), dove: \\
\( V={ v\ped{1}, ...,v\ped{n}} \) è un insieme finito di nodi o vertici. \\
L'insieme A di archi è un insieme di coppie \emph{ordinate} di vertici. \\
Un arco ha un senso di percorrenza(che si indica con una freccia), dalla testa alla coda. \\

Nodo iniziale e finale di uno arco $=$ coda e testa dell'arco
\begin{itemize}
\item \textbf{Archi paralleli:} hanno la stessa testa e la stessa coda.
\item \textbf{Cappio:} la testa e la coda coincidono.
\end{itemize}
\textbf{Grafo orientato Semplice:} non ci sono cappi e nemmeno archi paralleli.
\textbf{Multigrafo orientato:} ci sono cappi e archi paralleli. 

\subsubsection{Terminologia}
\textbf{Cammino orientato:} sequenza di nodi distinti, dove ogni coppia di nodi consecutivi nel cammino è
collegata da un arco (nodo iniziale, nodo finale) \\
\textbf{Lunghezza del cammino:} numero di archi nel cammino. \\
Un grafo orientato è \textit{fortemente connesso} se per ogni coppia di nodi $u$, $v$ esiste un cammino orientato da $u$ a $v$ ed un cammino orientato da $v$ a $u$. \\
\textbf{Circuito:} cammino nel quale il primo nodo coincide con l'ultimo. \\
\textbf{Lunghezza del circuito:} numero di archi. \\
\textbf{Parità del circuito:} parità della lunghezza.

\subsubsection{Proprietà dei grafi orientati}
\textbf{\textit{Semplice} massimo numero di archi}: numero di coppie ORDINATE di \(n=\mid V \mid \) elementi. \\
\textbf{In-degree(v)} $=$ Grado entrante di un nodo v: numero di archi entranti in v \(\Rightarrow d\ap{in}(v)\) \\
\textbf{Out-degree(v)} $=$ Grado uscente di un nodo v: numero di archi uscenti da v \( \Rightarrow d\ap{out}(v) \)\\

Per ogni grafo orientato D(V,A) vale:
 \[ \mid A \mid = \sum{v \in V} d\ap{in}(v) = \sum{_v \in V} d\ap{out}(v) \] 

\paragraph{Teorema} In ogni grafo orientato D(V,A), sono uguali tra loro:
\begin{itemize}
\item la somma dei gradi uscenti dei nodi
\item la somma dei gradi entranti dei nodi
\item il numero di archi del grafo
\end{itemize}

\begin{equation}
    \sum{v \in V} d\ap{out}(v) = \sum{v \in V} d\ap{in}(v) = \mid A \mid 
\end{equation}

Un arco uv un contribuisce 1 a d\ap{out}(v) e 1 a d\ap{in}(v)

\paragraph{Tornei:} Un \textit{torneo} è un grafo orientato in cui per ogni coppia di vertici esiste \textsc{esattamente} uno degli archi uv , uv. \\
(Si chiamano tornei (tournaments) perche indicano gli esiti di un torneo e il senso di ogni arco indica l'esito dell'incontro fra i giocatori rappresentati dai suoi estremi).

\subsubsection{Rappresentazione di grafi mediante matrici}
\begin{enumerate}
\item \textsc{Matrice di incidenza vertici/archi}
\item \textsc{Matrice di incidenza nodi/archi}
\item \textsc{Matrice di adiacenza vertici/vertici o nodi/nodi}
\item List of edges, linked adjacency list, forward star, backward star ect..
\end{enumerate}

\newpage

\section{Terza lezione: Connettività}

Ricordiamo: dato G(V,E) u,v sono connessi se esiste un cammino con estremità u, v. \\
La CONNESSIONE ha le seguenti proprietà:
\begin{itemize}
\item u è connesso a se stesso . \( \rightarrow \) \emph{riflessività}
\item u,v connessi, v,u connessi. \( \rightarrow \) \emph{simmetria} 
\item u,t connessi, t,v, connessi, allora u,v connessi \( \rightarrow \) \emph{transitività}
\end{itemize}
Queste proprietà implicano che esiste una \emph{partizione} di V in parti \(V1,...Vk \) dove: \\
u,v sono connessi se appartengono alla stessa parte. \\
Queste parti sono le COMPONENTI CONNESSE di G. \\
G è \emph{connesso} se c'è una sola parte! \((k=1)\). \\

Dato \( S \subseteq V \) definiamo il \emph{taglio} associato ad S come l'insieme di archi. \\
\( \delta(S) = \{ \{uv\} \in E , \mid S \cap \{u, v\} \mid = 1\} \) \\
e diciamo che $\delta(S)$ \emph{separa} $u,v$ se \( \mid S \cap u,v \mid =1 \) \\

\paragraph{Teorema} 
Dato G(V,E), vertici $u,v$ appartengono alla stessa componete connessa di G se e solo se non esiste un taglio $\delta(S) = \emptyset $
Dimostriamo che: dato $\delta(S)$ che separa $u,v$ e $P$ cammino fra $u,v$ allora $P$ e $\delta(S)$ hanno almeno un arco in comune, cioè \(\mid P \cap \delta(S) \mid \geq 1\)\\
Se $\delta(S) \neq \emptyset$ per ogni taglio $\delta(S)$ che separa $u,v$ allora esiste un cammino fra $u,v$. \\

\subsection{Calcolo delle componenti connesse}

\textbf{Input: }G(V,E), v in V \\
\textbf{Output:} Componente connessa C che contiene v \\
\begin{itemize}
\item Poni $v \in C$ e dichiara $v$ non esaminato
\item \emph{esame vertice:} scegli $x \in C$ non esaminato. Aggiungi a $C$ ongi vertice adiacente a $x$ che non è in $C$.
\item \emph{stop} quando ogni vertice in $C$ è esaminato.
\item $C$ è la componente connessa che contiene $v$.
\end{itemize}
Sia $C$ componente connessa che contiene $v$ calcolata dall'algoritmo. Si noti che \(\delta(S) = \emptyset \)

\subsection{Connettività}
Ricordiamo: un grafo non orientato G(V,E) si dice \emph{connesso} se, per ogni coppia di vertici, esiste
un cammino che li collega, altrimenti si dice \emph{disconnesso}. \\
Le componenti connesse di un grafo sono i suoi sottografi connessi massimali. \\

\paragraph{Connettività sugli archi} Arco connettività fra \( u,v = K^E_{uv} (G)\) \\
Cardinalità minima di un taglio che separa \( u,v = min \{ \mid \delta(S) \mid : S \) separa \( u,v \}\) \\
Ricordiamo: dato G(V,E) u,v in componenti connesse distinte se e solo se \(K^E_{uv} (G) =0\) \\
Conseguenza: \(K^E_{uv} (G)\) è il minimo numero di archi da togliere a G affinchè u,v diventino disconnessi. \\

\paragraph{Teorema} Per ogni \( u,v K^E_{uv} (G)\) è il massimo numero di cammini con estremità $u,v$ che non hanno archi in comune (disgiunti sugli archi). \\

Ricordiamo che: G(V,E) connesso se e solo se non esiste \( \delta(S)\) che separa due vertici tali che \( \delta(S) = \emptyset \) \\

Definiamo: arco connettività di G $K^E(G) =$ minimo numero di archi da togliere affinchè G sia disconnesso $=$ cardinalità minima di un taglio che separa almeno 2 vertici. \\
Allora: \( K^E(G) = \min K^E_{uv} (G)\)\\

\subsubsection{Connettività (sui vertici)}

Dato G(V,E) grafo connesso, semplice e non completo: \\
\emph{Connettività (sui vertici) K(G)} $=$ minimo numero di vertici la cui rimozione trasforma G in un grafo
sconnesso (= non connesso). \\

Vertex cutset = insieme di vertici $U \subseteq V $con le due seguenti proprietà:
\begin{itemize}
\item la rimozione di tutti i vertici di U trasforma G in un grafo sconnesso;
\item la rimozione di alcuni dei vertici di U, ma non tutti, lascia il grafo connesso.
\end{itemize}

Quindi, se G non è completo: \(K(G) = \) cardinalità del vertex cutset di G di cardinalità minima.\\
Un grafo completo con n vertici non ha vertex cutsets.
Per convenzione, la connettività sui vertici di un grafo completo con n vertici è posta uguale a n-1.\\

\subsubsection{Disequazioni}
\paragraph{Teorema} Sia G(V,E) grafo con almeno due vertici. Allora:
\[ K(G) \leq K^E(G) \leq \delta(G) \]
dove il simbolo d(G) indica il grado del vertice di grado minimo di G: \(\delta(G) = \min {d(v) v \in V} \)  \\
In particolare, se G è il grafo completo con n vertici, $n \geq 2$, allora \(K(G) = K^E(G) = \delta(G) =n-1\) \\
Se $G = K_n$ allora \(K(G) = n-1 \leq  K(G) \leq \delta(G) = n-1\), allora \(\lambda(G) =n-1 \) \\

\subsection{Affidabilità di una rete}

Rete di comunicazione tra soggetti: è necessario fornire percorsi alternativi di comunicazione, per fronteggiare:
\begin{itemize}
\item inattività di un soggetto;
\item guasto di una linea;
\item carico eccessivo di una linea, che superi la sua capacità.
\end{itemize}

Dato G grafo:
\begin{itemize}
\item per ogni coppia di nodi esistono almeno Ke(G) cammini alternativi che li collegano (tali che non ci
siano due cammini che usano qualche arco in comune);
\item per ogni coppia di nodi esistono almeno K(G) cammini alternativi che li collegano (tali che non ci
siano due cammini che usano qualche vertice in comune).
\end{itemize}
In generale vale la seguente formula: \\
\[K(G) \leq Ke(G) \leq d(G) \leq 2 \frac{ \mid E \mid}{\mid V \mid} \] grado medio in $G =$ media \[ \{ gr(v) : v \in V \} = \frac{\sum(v \in V) gr(v)}{ \mid V \mid} = \frac{2 \mid E \mid}{\mid V \mid} \] \\
Si dice che un grafo ha connettività ottima se vale \[ K(G) = Ke(g) = d(G) = 2 \frac{\mid E \mid}{\mid V \mid} \] \\
Proprietà:
\begin{enumerate}
\item ha massima connettività sugli archi e sui vertici tra tutti i grafi con $\mid V \mid$ vertici e $\mid E \mid$ archi
\item è $\delta$ -regolare
\end{enumerate}

\(K^E_{uv} (G) =\) numero massimo di cammini con estremità $u,v$ che non hanno archi in comune. \\
\(K_{uv} (G) =\) numero massimo di cammini con estremità $u,v$ che non hanno nodi intermedi in comune. \\











\newpage


\section{Quarta lezione: Grafi bipartiti e isomorfi}
\subsection{Grafi bipartiti}

Ricordiamo che: un grafo G(V,E) è bipartito se il suo insieme di vertici può essere partizionato in due
sottoinsiemi, $V_1$ e $V_2$, in modo che ogni arco ha un estremità in $V1$ e l'altra in $V2$. Scriveremo G($V_1$, $V_2$; E). \\
Massimo numero di archi in un grafo bipartito semplice: \( \mid V_1 \mid * \mid V_2 \mid \) 

\subsubsection{Condizione necessaria E sufficiente} 
\textbf{Osservazione 1:} Se G(V,E) è bipartito e G'(V',E') sottografo di G, allora G'(V',E') bipartito. \\
Ovvero:
G'(V',E') sottografo di G(V,E), G' è nonbipartito, allora anche G è nonbipartito. \\
\textbf{Osservazione 2:} Se G(V,E) è bipartito con bipartizione V1 e V2, sia v un vertice. Posso assumere senza perdita di generalità v in V1. Allora tutti I vertici adiacenti a v DEVONO stare in V2. \\
Cioe' se v in V1, allora l' arco vu "forza" u in V2.\\
\textbf{Osservazione 3:} Se G(V,E) contiene un ciclo dispari come sottografo, allora G(V,E) \emph{non} è bipartito.\\
\paragraph{Teorema:} 
Un grafo G(V,E) è bipartito se e solo se G non contiene un ciclo di lunghezza dispari. \\
Per quanto detto sopra, se G contiene un ciclo dispari, G \emph{non} è bipartito. \\
Dimostriamo che se G \emph{non} contiene un ciclo dispari, allora G è bipartito. Scegliamo v in V. Sia:
\begin{itemize}
\item V1 l'insieme dei nodi di G raggiungibili da v con un cammino di lunghezza pari.
\item V2 l'insieme dei nodi di G raggiungibili da v con un cammino di lunghezza dispari.
\end{itemize}
\emph{N.B.} Se G è bipartito allora \(V1 \cap V2 \)  $= \emptyset $ e nessun arco di G ha entrambe le estremità in V1 o in V2.

\subsubsection{Algoritmo per bipartizione}
\textbf{Input:} Un grafo G(V,E) \\
\textbf{Output:} Se G è bipartito, una bipartizione R, B di G. \\
Per ogni componente connessa di G(V,E):
\begin{itemize}
\item Poni \(v \in B , R = \emptyset \)
\item \emph{esame vertice} $v \in B (R)$. Aggiungi a $R (B)$ ogni vertice adiacente a $v$ non in $R \cup B$
\item \emph{stop} quando ogni vertice di $R \cup B$ è esaminato
\item G(V,E) è bipartito se e solo se ogni arco $e$ ha un estremità ub $R$ ed una in $B$
\end{itemize}



\subsection{Grafi isomorfi}

Due grafi G(V,E) e G'(V',E') sono isomorfi se esiste una corrispondenza biunivoca (isomorfismo) tra i
vertici di V e quelli di V' tale che: \\
due vertici di V sono adiacenti in G se e solo se i corrispondenti vertici di V' sono adiacenti in G'. \\
\paragraph{Determinare se due grafi sono isomorfi:}
Per farlo bisogna cercare l'isomorfismo. \\
Se due grafi sono isomorfi:
\begin{itemize}
\item hanno lo stesso numero di \emph{vertici}
\item hanno lo stesso numero di \emph{archi}
\item hanno lo stesso numero di vertici con lo \emph{stesso grado}
\end{itemize}

\subsubsection{Condizioni necessarie}
Due grafi sono isomorfi se:
\begin{itemize}
\item hanno lo stesso numero di \emph{vertici}
\item hanno lo stesso numero di \emph{archi}
\item hanno lo stesso numero di vertici con lo \emph{stesso grado}
\item i complementari devono essere isomorfi
\item hanno gli stessi sottografi indotti.
\end{itemize}
Se le prime 3 condizioni sono verificate, costruisco il possibile isomorfismo accoppiando vertici dello
stesso grado, controllando che la condizione 5 via verificata. \\

%ESERCIZIO: Dimostrare che se G(V,E) soddisfa 3), allora G soddisfa anche 1) e 2)

%1) Sia G(V,E) un grafo 2-regolare. Dimostrare che ogni componente connessa di $G$ e' un ciclo.
%Sia G(V,E) un grafo con
%Dimostrare che ogni componente connessa di G e' un
%cammino o un ciclo. (un nodo isolato e' considerato un cammino di lunghezza 0).
%2) Siano G1 e G2 due grafi 2-regolari. Date un algoritmo per testare se sono isomorfi.
%3) In grafo seguente e' bipartito? (applicate l'algoritmo)






\newpage


\section{Quinta lezione: Foreste, alberi e cammini}
\subsection{Definizioni}
Una \emph{foresta} è un grafo senza cicli (aciclico). \\
Un \emph{albero} è una foresta connessa. \\

\noindent
Se F(V,E) è una foresta, allora $\mid E\mid = \mid V\mid-  n$ (con n numero delle componenti connesse di F) \\
Se T(V,E) è un albero, allora $\mid E\mid = \mid V\mid -1$. \\

\noindent
Se T(V,E) è un albero con almeno 2 nodi, allora T contiene almeno 2 nodi di grado 1. \\
\paragraph{Teorema} Sia T(V,E) un grafo connesso. Allora le seguenti affermazioni sono equivalenti:
\begin{itemize}
\item T non ha cicli.
\item Per ogni coppia di vertici distinti, x e y, esiste un unico cammino che li collega.
\item Il grafo T è minimalmente connesso, cioè la rimozione di un qualsiasi arco lo disconnette,
cioè ogni arco e' un taglio.
\end{itemize}
\textbf{DImostrazione:} \\
\(a \Rightarrow c \Rightarrow b \Rightarrow a \) \\

\noindent
\(a \Rightarrow c\) ) T connesso, senza cicli $\rightarrow$ ogni arco è un taglio. Suppongo che esista un arco $u,v$ tale che $T \setminus uv$ sia connesso. In $T \setminus uv$ esiste un cammino $P\ped{uv}$ che collega $u$ e $v$.\\
% cose a caso indecifrabili
$c \Rightarrow b$ ) T connesso tale che ogni arco è un taglio. \(\forall x, y \in V , x \neq y\) esiste un unico $P\ped{uv}$. \\
Suponiamo che esistano $x, y \in V, x \neq y$ tali che ci siano 2 cammini differenti che li collegano ($P' \ped{xy}$ e $P''\ped{xy}$), sia (u,v) il primo arco di $P\ped{xy}$ che non appartiene a $P''\ped{xy}$ \\
$T \setminus (uv)$ non è connesso, u e v appartengono a differenti componenti connesse. \emph{ma} in $T \setminus (uv)$ esiste un persorso u - v. \\
$b \Rightarrow a$ ) T tale che \(\forall x,y \in V, x \neq y\) esiste un unico cammino.
Supponiamo che C sia un circuito di T. Siano x e y $\in$ C, x $\neq$ y. \\
Esistono due cammini distinti che collegano x e y.
\\

\subsection{Albero di peso minimo. Algoritmo di Kruskal.}
\textbf{Problema:} Dato G(V,E) connesso, con pesi \( w(e), e \in E\) \\
\textbf{input:} G(V,E) connesso, pesi \\
\textbf{output:} Albero di peso minimo.\\
\begin{itemize}
\item \emph{Ordina} gli archi ${1, 2, .., m}$ in ordine nondecrescente di peso: \( w(1)\leq w(2)\leq .. \leq w(m)\)
\item \emph{Inizializzazione:} Poni $T=\emptyset$ ed $i=1$.
\item \emph{Iterazione i:} Se l'arco i ha le estremità in componenti connesse distinte di T, allora poni
\(T \leftarrow T \cup \{i\} \). Altrimenti \(T \leftarrow T\). Poni \(i \leftarrow i+1\) e ripeti l'iterazione.
\item \emph{Stop:} Quando $i=m$.
\end{itemize}

\subsection{Esercizi}
\subsubsection{Es.1}
Dimostrare che ogni grafo non orientato semplice G(V,E), senza cicli e con \(|E|=|V|-1\) è un albero. \\
\textbf{Svolgimento:} \\
Devo dimostrare che G(V,E) è connesso. Sia $c$ il numero di componenti connesse di G. \\
\(G_1(V_1,E_1)\) connesso e senza circuiti $\rightarrow$ albero; \( |E_1| = |V_1| -1 \) \\
\(G_c(V_c,E_c) \)connesso e senza circuiti $\rightarrow$ albero; \( |E_c| = |V_c| -1 \) \\
Sommando: \(|E|=|E_1|+ |E_2| + .. + |E_c| = |V_1|-1 + |V_2|-1 + .. + |V_c|-1 = |V| - c\) \\
Ma: \(|E| = |V| -1\) per ipotesi $\Rightarrow c=1 \Rightarrow$ G connesso $\Rightarrow$ è un albero.
%.....

\subsubsection{Es.2}
Dimostrare che ogni grafo non orientato semplice e connesso con $n$ vertici e $n-1$ archi,
è un albero.
%....

\subsubsection{Es.3}
Dimostrare che data una foresta F(V,E), vale: \(\mid E \mid = \mid V \mid - \gamma \) , con $\gamma =$ numero di componenti connesse di F(V,E).
%...

\subsubsection{Es.4}
\textbf{Problema:} Dato un grafo connesso con lunghezze positive sugli archi, e vertici r, v,
\emph{trovare} un cammino di lunghezza minima fra r e v. (lunghezza di un cammino $P =$ somma delle
lunghezze degli archi in P, lunghezza minima fra r e $v=$ distanza fra r e v). \\
Uso l'algoritmo di \emph{Dijkstra}. \\
\textbf{Input}: G(V,E) connesso, lunghezze \(l(e) > 0, e \in E\), non di partenza r \\
\textbf{output:} distanze (e cammini minimi) fra r e gli altri nodi in V.\\
\emph{Inizzializzazione:} Poni d(r) $=$ 0, d(v)$= \inf$, \(v \in V \setminus \{r \} i= 0, S_0 = \{r \}\). \\
\emph{Iterazione i:} \(\forall v \in V \setminus S_i\) poni \(d(v)= \min \{ d(v), d(u)+l(uv) \}\). \\
Sia $v \in V \setminus S_i$ il vertice per cui d(v) è minimo. \\
Poni \(i=i +1, S_i=S\ped{i-1} \cup \{ v\}\). \\
\emph{Stop:} Quando $i= |V-1|$.
%...


\newpage


\section{Sesta lezione: Grafi planari}
Un grafo non orientato si dice \emph{planare} se lo si può disegnare sul piano senza intersezioni tra gli archi. \\
Tale disegno viene detto \emph{rappresentazione piana} del grafo.\\

\noindent
Come stabilire se un grafo è planare? \\
Problema polinomiale, anche se l'algoritmo è complicato. Ne daremo una versione semplificata, chiamata il \emph{metodo del cerchio e delle corde}.

\subsection{Minori}
Contrazione di un arco u,v. Identifico u, v in un unico vertice. Arco u,v è diventato
un cappio ( e viene usualmente rimosso.) \\
G'(V',E') è minore di G(V,E) se G' è ottenuto da G tramite:
\begin{itemize}
\item contrazione di archi
\item rimozione di archi
\item rimozione di vertici isolati
\end{itemize}
\textbf{Teorema:} G planare, G' minore di G, allora G' planare. \\
Ognuna delle 3 operazioni applicata ad una rappresentazione piana di G, mantiene piana la rappresentazione.

\subsection{Teorema di Kuratowski}
Un grafo è planare se e solo se non contiene $K\ped{3,3}$ o $K_5$ come minore.\\

\noindent
\subsection{Disegni equivalenti per grafi planari:} Dato un grafo planare, ci sono molte sue rappresentazioni piane, tutte equivalenti. \\
Ogni rappresentazione piana divide il piano in \emph{regioni} o \textbf{facce}. Indichiamo con \emph{r} il numero di queste
regioni. Il numero r dipende solo dal grafo, non dalla particolare rappresentazione piana disegnata. 

\subsubsection{Metodo del cerchio e delle corde}
Dato un grafo G(V,E) non orientato:
\begin{enumerate}
\item trovare, se esiste, un circuito di lunghezza massima ( che
contenga tutti i vertici del grafo, lo chiameremo circuito
Hamiltoniano)
\item disegnare questo circuito come un grande cerchio
\item scrivere l'elenco degli archi del grafo che non sono
contenuti nel circuito; li chiamiamo corde e li inseriamo
internamente o esternamente al cerchio, cercando di
evitare gli incroci, seguendo i passi 4 e 5
\item scegliamo una corda e inseriamola, ad esempio,
internamente al cerchio, togliendola dall'elenco;
\item Tutte le corde che la incrociano debbono essere inserite
esternamente. Se la rappresentazione e' piana,
proseguiamo. Altrimenti il grafo non e' planare.
\end{enumerate}
\emph{Note:}
\begin{itemize}
\item Attenzione all'ordine con il quale si inseriscono le corde:
non possiamo fare scelte, solo la prima corda può essere
posta internamente o esternamente a nostra scelta.
\item Questa procedura non è sempre applicabile; in
particolare, non è applicabile se il grafo non contiene un
circuito hamiltoniano.
\end{itemize}

\subsubsection{Esempi}

\subsection{Formula di Eulero}
Se G è un grafo non orientato, connesso e planare, allora ogni sua rappresentazione piana ha r facce con:
\[r = e -v + 2\]
Con \(v = |V| = \) numero di vertici,
\(e = |E| =\) numero di archi,
\(r =\) numero di facce (regioni). \\

\noindent
\textbf{Dimostrazione:}

\subsection{Politopi convessi}
\paragraph{Formula di Eulero per i politopi}
Sia P un politopo convesso nello spazio tridimensionale. \\
Sia \(v =\) numero dei vertici,
\(e =\) numero degli archi,
\(r =\) numero delle facce,
allora vale la relazione \(r = e - v + 2\) . \\
\textit{Esempio:} un cubo ha \(r=6, e=12 v=8 \Rightarrow e- v+2=12-8+2=6=r\)
Ogni politopo si può trasformare in un grafo connesso e planare.

\subsubsection{Corollari}
\paragraph{Corollario 1:} Se G(V,E) è un grafo semplice e planare, allora \(e \leq 3v - 6\). \\
\textbf{Dimostrazione:}

\paragraph{Corollario 2:} Se G(V,E) è un grafo planare semplice, connesso e bipartito, allora \(e \leq 2v - 4\).\\
\textbf{Dimostrazione:}

\paragraph{Corollario 3:} Se G(V,E) planare, semplice e G contiene un vertice di grado \(<= 5\) non posso applicare la formula \(e<=3v-6\) perchè se ogni nodo ha grado $>5$ (cioè $\geq 6$) la formula \(\sum \ped{v \in V} d(v) = 2 |E| \) crea contraddizione, infatti: \(6v = 2|E|\) ossia \(3v = e\). 

\subsection{Problema dei 4 colori}
\textbf{Problema:} quanti colori sono necessari per colorare i paesi di una
mappa in modo tale che paesi adiacenti abbiano colori diversi?
\begin{center}
Mappa \\
\(\Downarrow \) \\
Grafo planare, ad ogni paese corrisponde un vertice.
\end{center}
\textbf{Problema:} quanti colori sono necessari per colorare i vertici di un grafo
planare, in modo che vertici adiacenti abbiano colori diversi? \\

\noindent
\textbf{Congettura dei 4 colori, dimostrata da Appel e Haken (1976):} tutti i grafi planari sono 4-colorabili.


\newpage


\section{Settima lezione: I circuiti hamiltoniani}
\subsection{Circuiti hamiltoniani}
Dato G(V,E) grafo, un (circuito) ciclo Hamiltoniano di G è un ciclo che visita \textbf{ogni vertice} esattamente
\textbf{una volta}. \\
Il grafo G viene detto Hamiltoniano se ha un ciclo Hamiltoniano. \\
Ricordiamo che abbiamo visto un algoritmo per testare la planarità di un grafo, conoscendo un ciclo
Hamiltoniano. \\
Stabilire se un grafo è Hamiltoniano è un problema difficile (NP-completo): appartiene ad una classe di
problemi per cui non esistono algoritmi semplici e veloci che risolvano qualsiasi istanza del problema. \\
Esistono condizioni \emph{necessarie} (che ogni grafo deve soddisfare per essere Hamiltoniano) e \emph{sufficienti} (che assicurano che un grafo che le soddisfi sia Hamiltoniano, ma non viceversa.) ma non esistono condizioni semplici che siano \emph{necessarie} e \emph{sufficienti} (contemporaneamente).

\subsubsection{Condizioni necessarie}
Osserviamo che: ogni condizione che è verificata da un ciclo Hamiltoniano H deve essere verificata da un grafo
che contiene H. \\
Se G(V,E) è Hamiltoniano, allora:
\begin{itemize}
    \item \(d(v) \geq 2  \forall v \in V\)
    \item \(K(G) \geq 2 \)
    \item Più \emph{importante}: per ogni sottoinsieme S di V, abbiamo \( |S| \geq \) delle componenti connesse di \(G \setminus S\).
\end{itemize}

\subsubsection{Condizioni sufficienti per l'esistenza di un ciclo hamiltoniano}
\paragraph{Teorema di Dirac} Sia G(V,E) un grafo semplice, con n vertici, \(n>2\), se \(d(v) \geq n/2\) per ogni vertice \(v \in V\), allora il grafo G è Hamiltoniano. \\
La condizione di Dirac è \emph{sufficiente} ma \emph{\textbf{non} necessaria}.

\subsection{Regole per costuire un ciclo Hamiltoniano}
Dato G (V,E), le seguenti regole possono aiutare nella ricerca di un ciclo hamiltoniano (CH), se
esiste. \\
Se \( K(G) \leq 1\), allora G non contiene nessun CH.\\
\begin{enumerate}
    \item Se un vertice ha grado 2, allora i due archi incidenti in esso devono appartenere a qualsiasi CH.
    \item Un ciclo Hamiltoniano non può contenere sottocicli propri.
    \item Se nella costruzione di un CH abbiamo individuato due archi incidenti nel vertice v e
    appartenenti al CH, allora tutti gli altri archi incidenti in v non possono appartenere a CH e
    possone essere rimossi per la ricerca del CH.
\end{enumerate}
Ricordiamo che: dato G(V,E), un ciclo è \emph{hamiltoniano} se contiene
tutti i vertici di G. \\ 
G è \emph{hamiltoniano} se contiene un ciclo hamiltoniano.\\ 
%Grafi hamiltoniani: Kn, n>=3 Kn,n n>=2 ipercubi
%Grafi nonhamiltoniani: Kn,m n diverso da m, Petersen. Scacchiera.

\noindent
\subsection{Percorsi euleriani}
Ricordiamo che: dato G(V,E), un \emph{percorso} è una sequenza v1, e1, v2, e2, v3, e3 ... vn, en, vn+1 (con possibile
ripetizione di vertici o archi) dove \(ei=vi,vi+1\). \\ 
Il percorso è \textbf{chiuso} se \(v1=vn+1\).\\ 
Il percorso è \emph{euleriano} se è chiuso e contiene \emph{esattamente una volta} tutti gli archi di G(V,E) (ma i vertici
possono essere attraversati più di una volta). \\ 
G(V,E) è \emph{euleriano} se contiene un percorso euleriano. 
\paragraph{Teorema di Eulero} G(V,E) è euleriano se e solo se G è connesso ed ogni vertice di G ha grado pari. 
\subsubsection{Algoritmo per trovare un percorso euleriano}
Parti da un nodo
arbitrario e inizia a percorrere il grafo (rimuovendo gli archi già percorsi). Percorri un arco che non
appartiene ad un ciclo solo se è l'unico che puoi percorrere. \\





\newpage


\section{Ottava lezione: Calcolo combinatorio}
Ricordiamo dalla prima lezione: \textbf{contare} un numero (finito) di oggetti non è sempre facile ma è un
problema che si pone spesso in informatica: per esempio contare le operazioni fatte da un algoritmo per
misurarne l'efficenza. \\
Vediamo \emph{due principi fondamentali} nei metodi di conteggio. 

\subsection{Principio di addizione}
\textbf{Esempio 1:} In una scuola vengono offerti due corsi opzionali, uno di scacchi e uno di violino.
Il corso di scacchi viene frequentato da 30 studenti, quello di violino da 40.
Quanti studenti hanno scelto corsi opzionali? \\

\textbf{Principio di addizione:} Abbiamo m insiemi con le seguenti proprietà: 
\begin{itemize}
    \item Il primo insieme contiene $r_1$ oggetti differenti
    \item Il secondo insieme contiene $r_2$ oggetti differenti
    \item L' m-esimo insieme contiene $r_m$ oggetti differenti
\end{itemize}    
Gli insiemi sono disgiunti a due a due. \\
Allora il numero di modi per selezionare un oggetto da uno degli insiemi è: \[r_1 + r_2+ \dots  r_k \]

\subsection{Principio di moltiplicazione}
\textbf{Esempio 2:} Lanciamo due dadi, uno verde e uno rosso, e leggiamo l'esito di questa procedura
(ad esempio, un esito può essere "6 sul verde, 3 sul rosso").
\begin{enumerate}
    \item Quanti differenti esiti si possono avere?
    \item Quanti differenti esiti si possono avere che non siano doppie (stesso valore
    sui due dadi)?
\end{enumerate}
\textbf{Principio di moltiplicazione:} Abbiamo una procedura formata da m fasi successive ordinate,
con le seguenti proprietà:
\begin{itemize}
    \item la prima fase ha $r_1$ esiti differenti
    \item la seconda fase ha $r_2$ esiti differenti
    \item la m-esima fase ha $r_m$ esiti differenti
\end{itemize}
Il numero degli esiti di ogni fase è indipendente dalle scelte delle fasi precedenti, gli esiti complessivi della procedura sono tutti distinti. \\
Allora il numero dei differenti esiti della procedura è: \[ r_1  \times  r_2  \times \dots r_m\]

\paragraph{Esempio 3} In uno scaffale della biblioteca ci sono 11 libri in inglese (distinti),
7 in francese (distinti) e 4 in russo (distinti).
In quanti modi possiamo scegliere due libri (coppia non ordinata) che non siano
della stessa lingua?

\paragraph{Esempio 4} Abbiamo a disposizione le lettere a, b, c, d, e, f per formare sequenze di lunghezza 3.
Quante ne possiamo formare con le seguenti regole?
\begin{itemize}
    \item Le ripetizioni sono permesse.
    \item Le ripetizioni sono vietate.
    \item Le ripetizioni sono vietate; ci deve essere la lettera "e".
    \item Le ripetizioni sono permesse; ci deve essere la lettera "e".
\end{itemize}

\subsection{Permutazioni e combinazioni semplici} 
Dati n oggetti distinti: \(n \in N\) \\
\textbf{Permutazione:} disposizione \emph{ordinata} di questi n oggetti. \\
\textbf{R-permutazione:} disposizione \emph{ordinata} di questi n oggetti usando r degli n oggetti. \\
\textbf{R-combinazioni:} sottoinsieme o selezione \emph{non ordinata} di r degli n oggetti. \\

\subsection{Esempi}
\subsubsection{Es. 1} Partecipiamo ad una gara di corsa, siamo in 100 tutti ugualmente bravi,
arriviamo tutti al traguardo. \\ 
Quante sono le possibili classifiche in cui io sono terza?

\subsubsection{Es. 2} Quanti sono gli anagrammi della parola DETTATO, cioè quante sono le disposizioni delle sue lettere? \\ 
E in quanti di questi anagrammi le tre "T" sono consecutive?

\subsubsection{Es. 3} Quante sequenze binarie di lunghezza 8 ci sono con sei elementi uguali a "1" e due elementi uguali a "0"?

\subsubsection{Es. 4} Abbiamo a disposizione 7 donne e 4 uomini, tra loro dobbiamo scegliere un comitato di k persone. 
In quanti modi lo possiamo formare sotto le seguenti regole? 
\begin{itemize}
    \item Il comitato è formato da 3 donne e 2 uomini.
    \item Il comitato può avere qualsiasi numero di aderenti, ma il
    numero delle donne deve essere uguale al numero degli
    uomini.
    \item Il comitato è formato da 4 persone e una di loro deve essere il signor Rossi.
    \item Il comitato è formato da 4 persone, e almeno due sono donne.
    \item Il comitato è formato da 4 persone, 2 donne e 2 uomini, e il signor e la signora Rossi
    non possono appartenerci insieme.
\end{itemize}

\subsubsection{Es. 2-bis}
Quanti sono gli anagrammi della parola DETTATO, cioè quante sono le disposizioni delle sue lettere? \\ 
Quanti anagrammi hanno la lettera "D" che precede la lettera "E"? \\ 
E quanti anagrammi hanno la lettera "D" che precede la lettera "E" e le tre "T" in posizioni consecutive?



\newpage



\end{document}