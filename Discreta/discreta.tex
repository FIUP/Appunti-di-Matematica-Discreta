\documentclass[12pt,a4paper]{article}

\usepackage[english,italian]{babel} %lingue utilizzate
\usepackage[utf8]{inputenc}
\usepackage[T1]{fontenc}

\usepackage{graphicx} %img/media
\usepackage{enumitem} %lists
\usepackage{listings} %coding environment 
\usepackage{mathtools} %math package
\usepackage{amssymb} %pacchetto per i simboli di matematica
\usepackage{amsmath} %pacchetto di matematica
\usepackage{amsthm} %pacchetto di matematica

\usepackage{hyperref} %linkare l indice
\hypersetup{hidelinks} %nascondere i link dell'indice


\begin{document}
    \author{Sara Righetto}
    \title{Appunti di Matematica Discreta}
   % \includegraphics[width=0.50\textwidth]{img/logo.png}

    \maketitle

    \newpage
    \tableofcontents %indice

    \newpage

    \begin{abstract} 
        La matematica discreta, è lo studio di strutture matematiche che sono fondamentalmente discrete, nel senso che non
    richiedono il concetto di continuità che voi vedete nei corsi di analisi. \par
    La maggior parte degli oggetti studiati nella matematica discreta sono insiemi numerabili come i numeri interi o
    insiemi finiti, come gli interi da 1 a k. \par
    Altri oggetti importanti sono i grafi, che modellano reti di connessione. \\

        La matematica discreta è diventata famosa per le sue applicazioni in informatica. I concetti e le notazioni della matematica discreta sono utili per lo studio o la modellazione di oggetti o problemi negli algoritmi informatici e nei linguaggi di programmazione.
    \end{abstract}

\begin{center}
    \textit{Argomenti:}
    \begin{itemize}
        \item Teoria dei grafi 
        \item Metodi di conteggio
        \item Relazioni di ricorrenza
    \end{itemize}
\end{center}

    \newpage


%LEZIONI
\input{lezioni/Lez1.tex}

\input{lezioni/Lez2.tex}

\input{lezioni/Lez3.tex}

\input{lezioni/Lez4.tex}

\input{lezioni/Lez5.tex}

\input{lezioni/Lez6.tex}

\section{Settima lezione: I circuiti hamiltoniani}
\subsection{Circuiti hamiltoniani}
Dato G(V,E) grafo, un (circuito) ciclo Hamiltoniano di G è un ciclo che visita \textbf{ogni vertice} esattamente
\textbf{una volta}. \\
Il grafo G viene detto Hamiltoniano se ha un ciclo Hamiltoniano. \\
Ricordiamo che abbiamo visto un algoritmo per testare la planarità di un grafo, conoscendo un ciclo
Hamiltoniano. \\
Stabilire se un grafo è Hamiltoniano è un problema difficile (NP-completo): appartiene ad una classe di
problemi per cui non esistono algoritmi semplici e veloci che risolvano qualsiasi istanza del problema. \\
Esistono condizioni \emph{necessarie} (che ogni grafo deve soddisfare per essere Hamiltoniano) e \emph{sufficienti} (che assicurano che un grafo che le soddisfi sia Hamiltoniano, ma non viceversa.) ma non esistono condizioni semplici che siano \emph{necessarie} e \emph{sufficienti} (contemporaneamente).

\subsubsection{Condizioni necessarie}
Osserviamo che: ogni condizione che è verificata da un ciclo Hamiltoniano H deve essere verificata da un grafo
che contiene H. \\
Se G(V,E) è Hamiltoniano, allora:
\begin{itemize}
    \item \(d(v) \geq 2  \forall v \in V\)
    \item \(K(G) \geq 2 \)
    \item Più \emph{importante}: per ogni sottoinsieme S di V, abbiamo \( |S| \geq \) delle componenti connesse di \(G \setminus S\).
\end{itemize}

\subsubsection{Condizioni sufficienti per l'esistenza di un ciclo hamiltoniano}
\paragraph{Teorema di Dirac} Sia G(V,E) un grafo semplice, con n vertici, \(n>2\), se \(d(v) \geq n/2\) per ogni vertice \(v \in V\), allora il grafo G è Hamiltoniano. \\
La condizione di Dirac è \emph{sufficiente} ma \emph{\textbf{non} necessaria}.

\subsection{Regole per costuire un ciclo Hamiltoniano}
Dato G (V,E), le seguenti regole possono aiutare nella ricerca di un ciclo hamiltoniano (CH), se
esiste. \\
Se \( K(G) \leq 1\), allora G non contiene nessun CH.\\
\begin{enumerate}
    \item Se un vertice ha grado 2, allora i due archi incidenti in esso devono appartenere a qualsiasi CH.
    \item Un ciclo Hamiltoniano non può contenere sottocicli propri.
    \item Se nella costruzione di un CH abbiamo individuato due archi incidenti nel vertice v e
    appartenenti al CH, allora tutti gli altri archi incidenti in v non possono appartenere a CH e
    possone essere rimossi per la ricerca del CH.
\end{enumerate}
Ricordiamo che: dato G(V,E), un ciclo è \emph{hamiltoniano} se contiene
tutti i vertici di G. \\ 
G è \emph{hamiltoniano} se contiene un ciclo hamiltoniano.\\ 
%Grafi hamiltoniani: Kn, n>=3 Kn,n n>=2 ipercubi
%Grafi nonhamiltoniani: Kn,m n diverso da m, Petersen. Scacchiera.

\noindent
\subsection{Percorsi euleriani}
Ricordiamo che: dato G(V,E), un \emph{percorso} è una sequenza v1, e1, v2, e2, v3, e3 ... vn, en, vn+1 (con possibile
ripetizione di vertici o archi) dove \(ei=vi,vi+1\). \\ 
Il percorso è \textbf{chiuso} se \(v1=vn+1\).\\ 
Il percorso è \emph{euleriano} se è chiuso e contiene \emph{esattamente una volta} tutti gli archi di G(V,E) (ma i vertici
possono essere attraversati più di una volta). \\ 
G(V,E) è \emph{euleriano} se contiene un percorso euleriano. 
\paragraph{Teorema di Eulero} G(V,E) è euleriano se e solo se G è connesso ed ogni vertice di G ha grado pari. 
\subsubsection{Algoritmo per trovare un percorso euleriano}
Parti da un nodo
arbitrario e inizia a percorrere il grafo (rimuovendo gli archi già percorsi). Percorri un arco che non
appartiene ad un ciclo solo se è l'unico che puoi percorrere. \\





\newpage


\section{Esercizi svolti}


\paragraph{Es.1} 
Dimostrare che, se un grafo connesso e bipartito G(V1, V2; E) ha un ciclo hamiltoniano, allora
le cardinalità di V1 e di V2 devono essere uguali.
Inoltre, se un grafo bipartito ha un numero dispari di vertici, allora non può avere cicli
hamiltoniani. \\

\noindent
\textsc{Risposta:} Sia \(c = (x_1, x_2, \dots , x_n, x_1) \) un circuiti Hamiltoniano in G. \\
\(x_1 \in V_1 \Rightarrow x_2 \in V_2\) \\
\(x_3 \in V_1 \Rightarrow x_4 \in V_2\) \\
\(x_5 \in V_1 \Rightarrow x_6 \in V_2\) \\
$\dots$ \\
\(\exists (x_1,x_n) \in E \Rightarrow x_n \in V_2 \Rightarrow n\) pari \(\Rightarrow |V_1| = |V_2| \) \\
Se \(\exists c\) circuito Hamiltoniano $\Rightarrow n$ pari. 

\paragraph{Es.2}
\begin{enumerate}
    \item Quanti diversi cicli Hamiltoniani ci sono nel grafo \(K_n, n \geq 3\), il grafo completo con n vertici?
    \item Dimostrare che gli archi del grafo \(K_n\), con n primo, si possono partizionare in \(\frac{1}{2}(n-1)\) hamiltoniani disgiunti sugli archi.
    \item Se 11 persone cenano insieme ogni sera disponendosi attorno ad una tavola rotonda, e ogni
    sera ogni persona si vuole sedere tra due persone che non gli sono mai state vicino, quante cene
    riescono a fare?
\end{enumerate}
\textsc{Risposta:}

\paragraph{Es.3} Il seguente grafo è noto come grafo di Petersen.\\
E' un grafo con delle proprietà particolari e non è facile da studiare, pur avendo solo 10 vertici. \\
E' bipartito? Hamiltoniano? Planare? \\

\noindent
\textsc{Risposta:} \\
\textbf{Oss:} non è bipartito e non è hamiltoniano. \\
Non si può applicare il metodo del cerchio e delle corde perchè non c'è un ciclo hamiltoniano. \\
La disequazione \(e \leq 3v -6\) è soddisfatta. \\
Quindi possiamo solo cercare un $K \ped{3,3}$ (non può essere un $K_5$ perchè servirebbero 5 vertici non di grado 4). 

\paragraph{Es.4} Rispondere alle seguenti domande, motivando le risposte.
\begin{enumerate}
    \item Qual è il massimo numero di archi in un grafo non orientato semplice con 15 vertici?
    \item Qual è il massimo numero di archi in un grafo orientato semplice con 15 vertici?
    \item Qual è il minimo numero di vertici in un grafo non orientato semplice con 55 archi? E se ha 60 archi?
\end{enumerate}
\textsc{Risposta:}
\begin{enumerate}
    \item \( G(V,E)\)  \(|E| \leq \frac{|V|(|V|-1)}{2} = \frac{15*14}{2} =105 \)
    \item \(|E| \leq |V|(|V|-1) = 15*14=210\)
    \item \( G(V,E) |E|=55\) 
\end{enumerate}
Esiste un grafo completo $K_n$ con 55 archi? Se si, quanto deve valere n?\\

\noindent
\textsc{Risposta:} \\
\(|E(K_n)| = \frac{n(n-1)}{2}= 55\) \\
\(n(n-1)= 110\) \\
\(n^2 - n- 110=0\) \\
\(n= \frac{1 \pm \sqrt{1+440}}{2}= \frac{1 \pm \sqrt{441}}{2} = \frac{1 \pm 21}{2}=11 \) \\
Quindi $K\ped{11}$ ha 55 archi ed è il grafo con il minor numero di vertici tra quelli che hanno 55 archi. \\
Se \(|E|=60 \Rightarrow \) allora è necessario almeno un vertice in più \(\Rightarrow |V| \geq 12\). \\
Esistono grafi con 12 vertici e 60 archi? \\
\(|E(K\ped{12})| = \frac{12*11}{2}= 66 \Rightarrow\) quindi \(|V|= 12\)

\paragraph{Es.5} Qual è il massimo numero possibile di vertici in un grafo con 19 archi e tutti i vertici di grado maggiore o uguale a 3?\\

\noindent
\textsc{Risposta:}

\paragraph{Es.6} Se un grafo non orientato ha n vertici, tutti di grado dispari
tranne uno, quanti vertici di grado dispari ci sono nel suo complementare?\\

\noindent
\textsc{Risposta:}

\paragraph{Es.7} Determinare se i seguenti grafi sono bipartiti:\\

\noindent
\textsc{Risposta:}

\paragraph{Es.8} Rispondere alle seguenti domande, motivando le risposte.
\begin{itemize}
    \item Qual è il massimo numero di archi in un grafo non orientato e bipartito con 15 vertici?
    \item Qual è il minimo numero di vertici in un grafo non orientato e bipartito, con lo stesso numero di vertici nelle due parti della bipartizione, e con 65 archi?
    \item Qual è il minimo numero di vertici in un grafo non orientato e bipartito, con lo stesso numero di vertici nelle due parti della bipartizione, e con 64 archi?
\end{itemize}
\textsc{Risposta:}

\paragraph{Es.9} Dimostrare che un grafo con n vertici e più di \( \frac{n^2}{4}\) archi non può essere bipartito.\\

\noindent
\textsc{Risposta:}

\paragraph{Es. 10} Stabilire se i due seguenti grafi sono isomorfi, cercando di costruire l'isomorfismo. \\
%Imm grafi

\noindent
\textsc{Risposta:} sono isomorfi e l'isomorfismo è il seguente: 
\begin{itemize}
    \item \( a \longleftrightarrow 3\)
    \item \(e \longleftrightarrow 5\)
    \item \( d \longleftrightarrow 4\)
    \item \( f \longleftrightarrow 2\)
    \item \(c \longleftrightarrow  1\)
    \item \(b \longleftrightarrow 6 \)
\end{itemize}

\paragraph{Es. 11} Dimostrare che tutti i grafi (semplici ) con 5 vertici e con ciascun vertice di grado 2 sono
isomorfi. La stessa proprietà vale anche per grafi con 6 vertici?\\

\noindent
\textsc{Risposta:} Tutti i grafi con 5 vertici, con ciascun vertice di grado 2, sono circuiti di lunghezza 5($C_5$) e quindi tutti isomorfi tra loro.

\paragraph{Es. 12} Tracciare una rappresentazione piana dei seguenti grafi, dove ogni arco è un segmento retto
(notare che "e" è uno arco qualsiasi).\\

\noindent
\textsc{Risposta:}

\paragraph{Es. 13} Quale dei seguenti grafi è planare?\\

\noindent
\textsc{Risposta:} 

\paragraph{Es. 14} Stabilire quanti vertici deve avere un grafo planare connesso con 5 facce e con 10 archi.\\
Disegnare un grafo con queste caratteristiche.\\

\noindent
\textsc{Risposta:} Planare e connesso $\Rightarrow$ vale la formula di Eulero $\Rightarrow r=e-v+2$ \\
\(\Leftrightarrow 5=10-v+2 \Leftrightarrow v=7\)

\paragraph{Es. 15}Stabilire se può esistere un grafo semplice non orientato con le caratteristiche indicate. In caso affermativo darne un esempio, in caso negativo spiegare perché non può esistere.
\begin{enumerate}
    \item Esiste un grafo planare semplice con 9 vertici e 22 archi?
    \item Esiste un grafo planare semplice con 9 vertici e 21 archi?
    \item Esiste un grafo planare semplice e bipartito con 8 vertici e 13 archi?
    \item Esiste un grafo planare semplice e bipartito con 8 vertici e 12 archi?
\end{enumerate}
\textsc{Risposta:}
\begin{enumerate}
    \item Se esiste, allora deve valere \(e \leq 3v-6 \Leftrightarrow 22 \leq 3.9-6=21\), quindi non esiste.
    \item La disequazione \(e \leq 3v -6\) questa volta da $21<21$ vero, ma non è sufficiente per affermare che il grafo esiste, devo trovare un esempio.
    %grafo
    \item Se esiste allora deve valere \(e \leq 2v -4 \Leftrightarrow 13 \leq 2*8 -4= 12 \) quindi non esiste.
    \item La disequazione dà $12 \leq 12$, devo cercare un esempio.
    %grafo
\end{enumerate}

\paragraph{Es. 16} Per quali valori di n il grafo $K_n$ è planare? Per quali valori di $r$ e di $s$ il grafo completo bipartito $K_r,s$ è planare?\\

\noindent
\textsc{Risposta:} $K_n$ è planare $\Leftrightarrow n \leq 4$. \\
Dato che \(r=s=3 \) non è vero, e \(r \geq 3 , s \geq 3\) nemmeno, \(r=2\) e un $s$ qualsiasi si: è planare per $r\leq 2$ e $s$ qualsiasi o viceversa.

\paragraph{Es. 17} Considerare il grafo non orientato G(V,E) qui sotto disegnato e rispondere alle seguenti domande.
\begin{enumerate}
    \item Il grafo G è bipartito?
    \item Calcolare il grado minimo, il grado massimo e il grado medio dei vertici
    \item Percorrendo i vertici del grafo in ordine alfabetico, trovate un circuito Hamiltoniano; usatelo per
stabilire se il grafo G è planare. Se la risposta è affermativa, disegnate il grafo in modo piano; se
invece la risposta è negativa, fornite $K\ped{3,3}$ o $K_5$ come minore
    \item Determinare K(G) e Ke(G)
\end{enumerate}
\textsc{Risposta:} 

\paragraph{Es. 18} È possibile disegnare un grafo planare semplice con 4 vertici e 4 facce?
Se possibile disegnarlo.\\

\noindent
\textsc{Risposta:}

\paragraph{Es. 19} È possibile disegnare un grafo planare semplice con 5 vertici e 6 facce?
Se possibile disegnarlo.\\

\noindent
\textsc{Risposta:}

\paragraph{Es. 20} È possibile disegnare un grafo planare semplice con 11 vertici in cui ogni vertice abbia grado
maggiore o uguale a 5? \\

\noindent
\textsc{Risposta:}


\newpage

\section{Ottava lezione: Calcolo combinatorio}
Ricordiamo dalla prima lezione: \textbf{contare} un numero (finito) di oggetti non è sempre facile ma è un
problema che si pone spesso in informatica: per esempio contare le operazioni fatte da un algoritmo per
misurarne l'efficenza. \\
Vediamo \emph{due principi fondamentali} nei metodi di conteggio. 

\subsection{Principio di addizione}
\textbf{Esempio 1:} In una scuola vengono offerti due corsi opzionali, uno di scacchi e uno di violino.
Il corso di scacchi viene frequentato da 30 studenti, quello di violino da 40.
Quanti studenti hanno scelto corsi opzionali? \\

\textbf{Principio di addizione:} Abbiamo m insiemi con le seguenti proprietà: 
\begin{itemize}
    \item Il primo insieme contiene $r_1$ oggetti differenti
    \item Il secondo insieme contiene $r_2$ oggetti differenti
    \item L' m-esimo insieme contiene $r_m$ oggetti differenti
\end{itemize}    
Gli insiemi sono disgiunti a due a due. \\
Allora il numero di modi per selezionare un oggetto da uno degli insiemi è: \[r_1 + r_2+ \dots  r_k \]

\subsection{Principio di moltiplicazione}
\textbf{Esempio 2:} Lanciamo due dadi, uno verde e uno rosso, e leggiamo l'esito di questa procedura
(ad esempio, un esito può essere "6 sul verde, 3 sul rosso").
\begin{enumerate}
    \item Quanti differenti esiti si possono avere?
    \item Quanti differenti esiti si possono avere che non siano doppie (stesso valore
    sui due dadi)?
\end{enumerate}
\textbf{Principio di moltiplicazione:} Abbiamo una procedura formata da m fasi successive ordinate,
con le seguenti proprietà:
\begin{itemize}
    \item la prima fase ha $r_1$ esiti differenti
    \item la seconda fase ha $r_2$ esiti differenti
    \item la m-esima fase ha $r_m$ esiti differenti
\end{itemize}
Il numero degli esiti di ogni fase è indipendente dalle scelte delle fasi precedenti, gli esiti complessivi della procedura sono tutti distinti. \\
Allora il numero dei differenti esiti della procedura è: \[ r_1  \times  r_2  \times \dots r_m\]

\paragraph{Esempio 3} In uno scaffale della biblioteca ci sono 11 libri in inglese (distinti),
7 in francese (distinti) e 4 in russo (distinti).
In quanti modi possiamo scegliere due libri (coppia non ordinata) che non siano
della stessa lingua?

\paragraph{Esempio 4} Abbiamo a disposizione le lettere a, b, c, d, e, f per formare sequenze di lunghezza 3.
Quante ne possiamo formare con le seguenti regole?
\begin{itemize}
    \item Le ripetizioni sono permesse.
    \item Le ripetizioni sono vietate.
    \item Le ripetizioni sono vietate; ci deve essere la lettera "e".
    \item Le ripetizioni sono permesse; ci deve essere la lettera "e".
\end{itemize}

\subsection{Permutazioni e combinazioni semplici} 
Dati n oggetti distinti: \(n \in N\) \\
\textbf{Permutazione:} disposizione \emph{ordinata} di questi n oggetti. \\
\textbf{R-permutazione:} disposizione \emph{ordinata} di questi n oggetti usando r degli n oggetti. \\
\textbf{R-combinazioni:} sottoinsieme o selezione \emph{non ordinata} di r degli n oggetti. \\

\subsection{Esempi}
\paragraph{Es. 1} Partecipiamo ad una gara di corsa, siamo in 100 tutti ugualmente bravi,
arriviamo tutti al traguardo. \\ 
Quante sono le possibili classifiche in cui io sono terza?

\paragraph{Es. 2} Quanti sono gli anagrammi della parola DETTATO, cioè quante sono le disposizioni delle sue lettere? \\ 
E in quanti di questi anagrammi le tre "T" sono consecutive?

\paragraph{Es. 3} Quante sequenze binarie di lunghezza 8 ci sono con sei elementi uguali a "1" e due elementi uguali a "0"?

\paragraph{Es. 4} Abbiamo a disposizione 7 donne e 4 uomini, tra loro dobbiamo scegliere un comitato di k persone. 
In quanti modi lo possiamo formare sotto le seguenti regole? 
\begin{itemize}
    \item Il comitato è formato da 3 donne e 2 uomini.
    \item Il comitato può avere qualsiasi numero di aderenti, ma il
    numero delle donne deve essere uguale al numero degli
    uomini.
    \item Il comitato è formato da 4 persone e una di loro deve essere il signor Rossi.
    \item Il comitato è formato da 4 persone, e almeno due sono donne.
    \item Il comitato è formato da 4 persone, 2 donne e 2 uomini, e il signor e la signora Rossi
    non possono appartenerci insieme.
\end{itemize}

\paragraph{Es. 2-bis}
Quanti sono gli anagrammi della parola DETTATO, cioè quante sono le disposizioni delle sue lettere? \\ 
Quanti anagrammi hanno la lettera "D" che precede la lettera "E"? \\ 
E quanti anagrammi hanno la lettera "D" che precede la lettera "E" e le tre "T" in posizioni consecutive?



\newpage


\section{Decima lezione:}
Coefficenti binomiali: \[C(n,k) =  \binom{n}{k} \]
\\
\[ \binom{n}{k} = \frac{n!}{K!(n-k)!}\] se \( 0 \leq k \leq n\) interi.
\subsection{Triangolo di Pascal/Tartaglia}

\subsection{Identità binomiali} 
Verieficati in due modi diversi: 
\begin{itemize}
    \item espansione binomiale 
    \item significato combinatorio
\end{itemize}

\begin{enumerate}
    \item 
    \item 
    \item 
\end{enumerate}

\subsection{Esercizi}

\paragraph{Es. 1} Dimostrarze che la seguente identità binomiale vale per ogni n naturale,
usando l'interpretazione combinatoria dei coefficienti binomiali.
Verificarla anche tramite l'espansione algebrica dei coefficienti binomiali.

\paragraph{Es. 2} Dimostrare la seguente identità binomiale usando l'interpretazione combinatoria dei
coefficienti binomiali, dove m, n ed r sono numeri naturali, con \(0 \leq r \leq \min (m,n)\)

\paragraph{Es. 3} %esercizio 6
Risolvere la seguente equazione binomiale per n naturale, trovando tutte le sue soluzioni.

\paragraph{Es. 4} Dimostrate con metodi di conteggio oppure identità già viste le seguenti identità binomiali:

\paragraph{Es. 5} %eserc 3
Dimostrare la seguente identità binomiale, per n e k naturali con $0 \leq k \leq n$.

\paragraph{Es. 6}%eserc 5
Valutare le seguenti espressioni per n naturale (n diverso da zero per la prima):


\newpage


\section{Undicesima lezione: Disposizioni con ripetizione}

Le disposizioni con ripetizioni sono tutti i possibili ordinamenti di oggetti, alcuni dei quali indistinguibili. 

\noindent
\textbf{Esempio:} Quante sono le disposizioni delle lettere della parola BANANA? \\
Le lettere da assegnare ai 6 posti disponibili sono una B, due N e tre A. Avrei delle combinazioni quali
\smallskip
\( \binom{6}{1}, \binom{5}{2}, \binom{3}{3} = \frac{6!}{1! 5!} \frac{5!}{2! 3!} \frac{3!}{3! 0!} = \frac{6!}{1! 2! 3!}\)

\paragraph{Teorema} 
Abbiamo n oggetti, di cui
\begin{itemize}
    \item $r_1$ del tipo 1 (identici)
    \item $r_2$ del tipo 2 (identici)
    \item $\dots$
    \item $r_m$ del tipo m (identici)
\end{itemize}
con $r_1 + r_2 + ... + r_m = n$. \\
Il numero delle disposizioni di questi n oggetti è:\\
\smallskip
\( \binom{n}{r_1} \binom{n-r_1}{r_2} \binom{n-r_1-r_2}{r_3} \dots  \binom{r_m}{r_m} = \frac{n!}{r_1 ! r_2 ! \dots r_m !} \) \\
Numero di disposizioni con ripetizione 
%dimostrazione


\subsection{Selezioni con ripetizione}
\textit{Esempio:} In quanti modi possiamo selezionare 6 caffè tra le varietà: liscio, macchiato, lungo? \\


\paragraph{Teorema}
Il numero di selezioni con ripetizione di r oggetti scelti tra n varietà è: \\
\smallskip
\( \binom{r + n -1}{r}= \binom{r+n-1}{n-1}= \) numero di selezioni con ripetizione \\
= numero di soluzioni intere della equazione
\[ \left\{ 
    \begin{array}{rcl} 
    x_1 + x_2 + \dots + x_n = r \\
    x_1, x_2, \dots, x_n \geq 0
\end{array} 
\right. \]

\textit{Esempio} \\ %2 
$x_1 = $numero di caffè lisci \\
$x_2 = $numero di caffè macchiati \\
$x_3 = $numero di caffè lunghi \\
Le selezioni di 6 caffè tra 3 varietà sono tante quante le soluzioni intere della equazione

\paragraph{Esempio} %5
Quante sequenze di lunghezza 10 si possono formare scegliendo gli elementi tra
quattro "A",
quattro "B",
quattro "C",
quattro "D",
se ogni lettera deve apparire almeno due volte?

\paragraph{Esempio} %6
In una pasticceria, ci sono 5 varietà di dolci. In quanti modi possiamo scegliere 12
dolci tra queste 5 varietà, se vogliamo prendere almeno un dolce per varietà?

\subsection{Distribuzioni di oggetti distinti} 
r oggetti distinti, n scatole distinte: Quanti sono i modi di porre questi r oggetti nelle scatole, se
una scatola puo' contenere piu' di un oggetto?
 distribuzioni di r oggetti distinti in n scatole distinte
 stringhe di lunghezza r i cui elementi sono scelti in
{1, ... ,n} con ripetizione :

 distribuzioni di r oggetti distinti in n scatole distinte,
sapendo che nella scatola i devono entrare $r_i$ oggetti
($i=1, ... ,n$) con $r_1 + r_2 + ...+ r_n = r$
 disposizioni di r oggetti, di cui $r_1 $ del tipo 1, ........ ,

\subsection{Distribuzioni di oggetti identici}
r oggetti identici, n scatole distinte
 distribuzioni di r oggetti identici in n scatole diverse
 selezioni con ripetizione di r oggetti scelti tra n varietà

\paragraph{Esempio} %5
\begin{enumerate}
    \item Il numero di modi di distribuire r palline identiche in n scatole distinte.
Il numero di modi di distribuire r palline identiche in n scatole distinte,
con almeno una pallina per scatola ($r>=n$)
    \item Il numero di modi di distribuire r palline identiche in n scatole distinte,
con almeno $r_1$ palline nella prima scatola, almeno $r_2$ nella seconda, ... ,
almeno $r_n$ nella n-esima
\end{enumerate}

\paragraph{Esempio} %7
Vado al mercato con 5 euro da spendere. Un cestino di fragole costa 1.5 euro,
una banana o una mela o una arancia costano 50 cent. In quanti modi posso
spendere i miei soldi, supponendo di volerli spendere tutti?


\subsection{Esercizi}

\paragraph{Es.1} Quante diverse sequenze di lunghezza cinque si possono formare usando le lettere
A, B, C, D con ripetizione, con la condizione che le sequenze non contengano la parola BAD
(ad esempio, la sequenza ABADD è esclusa)?

\paragraph{Es.2} In quanti modi una persona può invitare un sottoinsieme di almeno 3 dei suoi 10 amici a cena?

\paragraph{Es.3} Quante sequenze di cinque caratteri (scelti tra le 26 lettere dell'alfabeto, anche ripetute)
contengono esattamente una A ed esattamente due B?

\paragraph{Es.4}
\begin{enumerate}
    \item Quante sequenze di dieci lettere si possono formare usando 5 differenti vocali
e 5 differenti consonanti (scelte tra le 21 consonanti)?
\item Tra queste, quante sono quelle che non hanno coppie consecutive di
consonanti e non hanno coppie consecutive di vocali?
\end{enumerate}

\paragraph{Es.5}
\begin{enumerate}
    \item Quanti diversi triangoli si possono formare congiungendo terne di vertici di un
ottagono (regolare)?
\item Quanti diversi triangoli si possono formare se si vieta di utilizzare terne che
contengano due vertici adiacenti nell'ottagono?
\item Quanti diversi triangoli si possono formare che abbiano due lati coincidenti con lati
dell'ottagono?
\item Quanti diversi triangoli si possono formare che abbiano un solo lato coincidente
con lati dell'ottagono?
\end{enumerate}

\paragraph{Es.6}
\begin{enumerate}
    \item Quanti numeri di sette cifre si possono formare con le cifre 3, 5 e 7?
\item Di questi numeri, quanti hanno tre cifre uguali a 3, due uguali a 5 e due uguali a 7?
\end{enumerate}




\newpage

\section{Dodicesima lezione: Esercizi}

\subsection{Esercizi}

\paragraph{Es. 7} Nove persone entrano in un bar per acquistare un panino per ciascuno.
Tre ordinano sempre il panino con il tonno,
due sempre quello con le uova,
e due sempre quello con il prosciutto.
Le rimanenti due persone ordinano ciascuna uno qualsiasi dei tre tipi di panino ordinati dagli altri.
Quante diverse selezioni non ordinate di panini sono possibili?

\paragraph{Es. 8}Ordinando le cifre 1, 2, 2, 4, 4, 6, 6 a caso, scriviamo un numero di sette cifre.
\begin{enumerate}
    \item Quanti diversi numeri maggiori di 3000000 (tre milioni) possiamo scrivere?
     \item Quanti numeri dispari maggiori di tre milioni possiamo scrivere?
\end{enumerate}

\paragraph{Es. 9}In quanti modi differenti si possono disporre le lettere della parola REPETITION
in cui la prima E compare prima della prima T?

\paragraph{Es. 10} In quanti modi si possono distribuire 36 caramelle identiche a quattro bambini:
\begin{enumerate}
    \item senza restrizioni
    \item in modo che ogni bambino riceva 9 caramelle
    \item in modo che ogni bambino riceva almeno una caramella
\end{enumerate}

\paragraph{Es. 11} In quanti modi si possono distribuire
7 identiche mele, 6 identiche arance e 7 identiche pere
fra tre persone diverse:
\begin{enumerate}
    \item senza restrizioni
    \item in modo che ogni persona riceva almeno una pera
\end{enumerate}

\paragraph{Es. 12} Dire quante sono le soluzioni intere della equazione
con le condizioni:

\paragraph{Es. 13} Dire in quanti modi si possono distribuire k palline in n scatole distinte (k<n) con al massimo
una pallina per scatola se:
\begin{enumerate}
    \item le palline sono distinte
    \item le palline sono identiche
\end{enumerate}

\paragraph{Es. 14}
In quanti modi si possono suddividere tra due scatole
4 palline rosse identiche, 5 palline blu identiche e 7 palline nere identiche, se:
\begin{enumerate}
    \item non ci sono restrizioni
    \item si vuole che nessuna scatola resti vuota
\end{enumerate}


\newpage

\section{Tredicesima lezione: Relazioni di ricorrenza}

Data una procedura con n oggetti, si vuole contare il numero di modi di eseguirla.\\
$a_n =$ numero di modi di eseguire la procedura con n oggetti ($n \geq 0$)\\
$a_0, a_1, a_2, \dots a_n, \dots $: successione di numeri da determinare.

\subsection{Relazioni di ricorrenza} 
La relazione di ricorrenza è una formula ricorsiva che esprime $a_n$ in funzione dei
precedenti termini della successione:
\[ \left\{ 
    \begin{array}{rcl} 
    a_n = f(a_0, a_1, a_2, \dots , a\ped{n-1}, n) \\
    a_0= \dots \\
    a_1 = \dots 
    \end{array} \right. \]    

In cui $a_0$ e $a_1$ sono condizioni iniziali.   \\
Soluzione di una relazione di ricorrenza $=$ formula esplicita per $a_n$ , che dipende solo da n.

\subsection{Esempi}

\paragraph{Es. 1} Permutazioni di n oggetti distinti

\paragraph{Es. 2} Contare il numero di modi di salire una scala con n gradini, se siamo in grado di fare passi da
uno o due gradini: \textit{relazione di Fibonacci}

\paragraph{Es. 3} Calcolare il numero di confronti necessari per trovare l'elemento
massimo di una sequenza di lunghezza n (supponendo che n sia una potenza
di 2).

\paragraph{Es. 4} Torre di Hanoi\\
\textit{Problema:} spostare la torre di dischi dal piolo A al piolo C.\\
\textit{Regole:}
\begin{itemize}
    \item si può spostare un solo disco alla volta;
    \item si può muovere un disco solo se è il primo della sua pila (quello più in alto);
    \item un disco spostato può essere appoggiato solo su un piolo vuoto o sopra un disco di
raggio maggiore.
\end{itemize}

\paragraph{Es. 5} Evoluzione di un capitale investito in banca con tasso di interesse 5\%.

\paragraph{Es. 6} I modi di selezionare k membri di un comitato presi fra n persone sono i modi di
selezionare il comitato in cui il sig. Rossi non è presente (k persone prese fra $n-1) + i$ modi di
selezionare il comitato in cui il sig. Rossi E' presente ($k-1$ persone prese fra $n-1$).
(Alla base della costruzione del triangolo di Pascal).

\subsection{Dividi e conquista}
Ordinamento di una lista di n numeri, o algoritmo di merge sort: data una serie di numeri, li divido a metà in modo ricorrente e ordino ogni parte divisa per poi riunirla alle altre.\\ 
\textit{Esempio:} \\
1 9 5 2    3 8 7 4 \\
$\Downarrow$  $\Downarrow$\\
1 2 5 9    3 4 7 8 \\
$\Downarrow$  $\Downarrow$\\
1 2 3 4 5 7 8 9 \\


\subsection{Relazioni lineari omogenee}

%esempii

\subsection{Relazioni non omogenee}

%esempi







\newpage

\section{Quattordicesima lezione: Relazioni dividi e conquista}
\( \left\{ 
    \begin{array}{rcl}
    a_n = c* a\ped{n/2} + f(n) \\
     \mbox{condizione iniziale }
    \end{array}
     \right. \)
\paragraph{Caso 1:} \(c =1, f(n)=d\) , \( a_n = a\ped{n/2}+ d\) \\
Soluzione generale: \(a_n = d [\log_2 n ]+ A\) con $n>0$ \\ 
Verifica: \\
\paragraph{Caso 2:}   \(c=2, f(n)  =d\) \\
Soluzione generale: \\
Verifica: \\

\subsection{Esempi ed esercizi}
\paragraph{Esempio: torneo di tennis} $a_n =$ numero di turni di un torneo di tennis con n giocatori, dove $n$ è una potenza di 2.





\newpage

\section{Quindicesima lezione: Esercizi}

\paragraph{Es.1} %esercizio 2
Trovare una relazione di ricorrenza per valutare il numero di coppie di conigli dopo n mesi
se:
\begin{itemize}
    \item inizialmente vi è solo una coppia di conigli appena nati
    \item ogni mese, ogni coppia di conigli che ha più di un mese genera una nuova coppia di
conigli
    \item nessun coniglio muore
\end{itemize}

\paragraph{Es.2} %esercizio 4
\begin{itemize}
    \item Trovare una relazione di ricorrenza per il numero di sequenze quaternarie (cioè ad elementi in {0, 1, 2, 3}) di lunghezza n
     \item con almeno un 1, con il primo 1 che precede il primo 0 (se la sequenza contiene almeno uno 0)
\end{itemize}    

\paragraph{Es.3} Trovare la relazione di ricorrenza per il numero delle sequenze ternarie
(0 1 2) che non contengono la sequenza "012"

\paragraph{Es.4} Quante sono le scacchiere contenute in una scacchiera n x n?


\newpage

\begin{center}
\textbf{Argomenti svolti:} \\

\textit{Prima parte del corso:}
\begin{center}
Grafi non orientati: Nozioni di base: percorsi, cammini, cicli, gradi, sottografi, sottografi indotti. Famiglie di grafi. \\
Grafi orientati: Nozioni di base: percorsi, cammini, cicli, gradi. Tornei \\
Grafi bipartiti, parità di cicli. Isomorfismo fra grafi. Algoritmi. \\
Arcoconnettività, Arcoconnettività fra 2 nodi, tagli .\\
Connettività, coconnettività fra 2 nodi, separatori . Relazioni fra connettività ed arcoconnettività. \\
Alberi di peso minimo. Algoritmo di Kruskal.\\
Cammini minimi. Algoritmo di Dijkstra.\\
Planarita: Rappresentazioni piane, facce, perimetri. Formula di Eulero.\\
Minori. K33 e K5 non planari. Teorema di Kuratowski.
\end{center}

\textit{Seconda parte del corso} 
\begin{center}
Cicli hamiltoniani, percorsi euleriani.\\
Capitolo 5: Metodi di Conteggio\par
5.1 Principio di addizione e moltiplicazione\par
5.2 Permutazioni, r-permutazioni, r-combinazioni (selezioni non ordinate)\par
5.3 Stringhe da alfabeto finito (=permutazioni di parole). Selezioni di n oggetti da k tipi (=ordini)\par
5.5 Identità binomiali.\\
Capitolo 7: Relazioni di ricorrenza\par
7.1 Modelli di relazioni di ricorrenza\par
7.2 Soluzioni di equazioni di ricorrenza "divide and conquer"\par
7.3 Soluzioni di relazioni di ricorrenza lineari omogenee\par
7.4 Soluzioni di relazioni di ricorrenza lineari non omogenee
\end{center}
\end{center}



\end{document}